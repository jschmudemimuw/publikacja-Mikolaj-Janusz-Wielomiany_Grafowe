\section{Introduction}
Monadic second-order transductions (\mso transductions) are transformations which input structures  and output  structures such as words, trees or graphs. The idea is that the output structure is defined in terms of the input structure by using \mso logic. \mso transductions were originally introduced to describe graph-to-graph transformations, see~\cite[p.~43]{arnborgLagergrenSeese1988}, \cite[Definition 6]{engelfriet1991}, \cite[Defintion 2.2]{courcelle1991}, and it is graph transformations that are the principal topic of the current paper. Nevertheless, \mso transductions have also recently seen a lot interest for simpler structures such as strings or trees. In~\cite[Theorem 13]{engelfrietMSODefinableString2001},  Engelfriet and Hoogeboom proved a seminal result which says  that functional \mso transductions  define exactly the same string-to-string functions as two-way deterministic transducers. This result made a connection between \mso transductions and the well-established literature on transducers. 

In the transducer literature, a central problem is equivalence: given two transducers, decide if they represent that same function. The equivalence  problem  is known to be decidable for two-way deterministic transducers~\cite[Theorem 1]{gurariEquivalenceProblemDeterministic1982}. Putting this together with the theorem of Engelfriet and Hoogeboom, we see that  equivalence is also decidable for string-to-string \mso transductions.
Similar decidability results for the equivalence problem have been obtained  for tree-to-tree and tree-to-string transductions.  The tree-to-string case is as general as the tree-to-tree case, because an output tree can be  injectively represented as a string using pre-order traversal, and injectivity of this encoding guarantees that the representation does not affect equivalence of transductions. In~\cite[Corollary 8.2]{seidlManethKemper2018}, Seidl, Maneth and Kemper show that equivalence is decidable for tree-to-string streaming transducers; the latter are shown by Alur and D'Antoni to be equivalent to \mso transductions in~\cite[Theorem 4.6]{alurStreamingTreeTransducers2017}. Putting these results together, we see that equivalence is decidable for tree-to-tree (and tree-to-string) \mso transductions.

The equivalence problem often becomes harder and more interesting  when the output structures have more (but not too many) automorphisms. Here is an example of this phenomenon: compare binary trees that have a sibling order with  binary trees that do not have a sibling order. A binary tree with sibling order can be encoded (via an \mso transduction) in a tree without sibling order, by using labels (or the tree structure if labels are not allowed) to differentiate between left and right children. The other direction does not work: there is no \mso transduction that injectively maps unordered trees to ordered trees. For these reasons, the equivalence problem for tree-to-tree transductions becomes harder when considering trees without a sibling order. Nevertheless, the latter problem is still decidable~\cite[Section 6]{boiretReducingTransducerEquivalence2018}, which is shown by encoding trees without a sibling order as polynomials in $\Int[x]$, and then applying the ``Hilbert method'', which is an algorithm using Hilbert's Basis Theorem that is described in~\cite{siglog2019} and based on~\cite{seidlManethKemper2018}.

Since  the equivalence problem gets more interesting for  structures with more automorphism, we propose in this paper to study the equivalence problem for transductions that operate on (finite) graphs, where automorphism are as general as they can be. Since we use \mso as our transduction formalism, and  \mso is undecidable on graphs of unbounded treewidth~\cite[Theorem 8]{seese1991structure}, we consider \mso transductions for graphs of bounded treewidth.

We do not know if the equivalence problem is decidable for \mso graph-to-graph transductions of bounded treewidth. We propose an approach to the problem -- which uses the Hilbert method of encoding combinatorial objects (graphs, trees, strings, etc.) as algebraic objects (numbers, polynomials, rational functions, etc.) In our specific case, we get some limited success by representing graphs as rational power series. This, together with a recent result  of Grohe, Dell and Rattan~\cite[Theorem 2]{groheDellRattan2018}, which describes a certain relaxation of isomorphism in terms of counting walks, gives us our main result, Theorem~\ref{thm:path-equivalence}, which says that equivalence is decidable for  graph-to-graph transductions of bounded treewidth, where output  graphs are equivalent modulo the relaxation of isomorphism from~\cite{groheDellRattan2018}.

We only scratch the surface of the equivalence problem for graph-to-graph transductions. We view this paper mainly as invitation to study the problem, together with some  basic infrastructure for its potential solution, such as register transducers and encodings that use algebra. It remains to be seen how helpful this infrastructure will be. We also highlight some special cases of the problem, which seem approachable, but are nevertheless still open. 


% (ideally, graph isomorphism) for every input graph, the output graphs produced by the two transductions are isomorphic.  we state a problem and propose a solution strategy. The problem is deciding  equivalence for graph-to-graph transductions, for graphs of bounded treewidth. The solution strategy is to model graphs as algebraic objects (polynomials, or rational power series) and to use algorithms from algebra (Hilbert's Basis Theorem, or Gr\"obner bases) to decide equivalence. 



% \begin{itemize}
%     \item {\bf Parameter.} A number $k \in \set{1,2,\ldots}$ and an equivalence relation $\sim$ on graphs. 
%     \item {\bf Input.}  Two mso transductions 
%     \begin{align*}
%         \xymatrix@C=3cm{
%     \text{graphs of treewidth $k$} 
%     \ar[r]^{f_1,f_2} &
%      \text{graphs of treewidth $k$}}
%     \end{align*}
%     \item {\bf Output.} Is $f_1$ equivalent to $f_2$ modulo $\sim$, i.e.~is it the case that
%     \begin{align*}
%     f_1(G) \sim f_2(G) \qquad \text{for every graph $G$ of treewidth $\le k$.}
%     \end{align*}
% \end{itemize}

% Our long term goal is to find out if the problem above is decidable or not  when  $\sim$ being graph isomorphism, and $k$ is arbitrary. In this paper, we show decidability for two special cases:
% \begin{itemize}
%     \item $\sim$ is isomorphism and $k=1$;
%     \item $\sim$ is a .. and $k$ is arbitrary.
% \end{itemize}


% \begin{lemma}
%     Given  mso transductions 
%     \begin{align*}
%         \xymatrix@C=3cm{
%     \text{graphs of treewidth $\le k$} 
%     \ar[r]^{f_1,f_2} &
%      \text{graphs of treewidth $\le k$}}
%     \end{align*}
%     one can compute mso transductions
%     \begin{align*}
%         \xymatrix@C=3cm{
%     \text{trees} 
%     \ar[r]^{g_1,g_2} &
%      \text{graphs of treewidth $\le k$}}
%     \end{align*}
% \end{lemma}