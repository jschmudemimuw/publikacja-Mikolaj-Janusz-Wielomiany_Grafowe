\section{Proofs for Section~\ref{sec:sec:equivalence-modulo}}

\begin{proof}[Proof of Lemma~\ref{lem:compute-power-series}]

    %J
    
        % The proof is an adaptation of a weighted version of the usual ''DFA to regular expression conversion'' (i.e. weighted automaton to rational power series), with state elimination corresponding to source forgetting.
    
        
        % Define an \emph{internal walk} to be a walk that does not visit any sources,  with the possible exception of the first and last vertices in the walk, which are allowed to be sources. 
        Define the \emph{source number} of a vertex in a $k$-sourced graph to be  $i \in \set{1,\ldots,k}$ if the vertex is the $i$-th source, and $0$ otherwise. For
        $i,j \in \set{0,\ldots,k}$,
        define an  $(i,j)$-walk to be a walk where the first vertex has source number $i$, the last vertex has source number $j$, and the remaining vertices in the walk are not sources. For a $k$-sourced graph $G$, define 
        \begin{align*}
        \varphi_{ij}(G) \in \aalg
        \end{align*}
         to be the power series where the $n$-th term is the number of nonempty $(i,j)$-walks of length $n$. The constant term  of this power series is zero, since we only consider nonempty walks. Define 
        \begin{align*}
        \varphi(G) \in \aalg^{\set{0,\ldots,k}^2}
        \end{align*}
        to be the tuple of all of the values returned by the functions $\varphi_{ij}$. 
        The main observation is that $\varphi$ is an algebra homomorphism, in the following sense.
        \begin{claim}
            For every term operation $f$ 
            in the algebra of $k$-sourced graphs (see the diagram below for its type), there is a term operation $f^\varphi$ in the algebra $\aalg$ (see the diagram for its type) which makes the following diagram commute
            \begin{align*}
            \xymatrix{
                (\text{$k$-sourced graphs})^X 
                \ar[r]^f
                \ar[d]_{\varphi^X}
                &
                 (\text{$k$-sourced graphs})^Y
                \ar[d]^\varphi
                \\
                \aalg^{\set{0,\ldots,k}^2 \times X}
                \ar[r]_{f^\varphi}
                &
                \aalg^{\set{0,\ldots,k}^2 \times Y}
            }
            \end{align*}
        \end{claim}
        Before proving the claim, we use it to finish the proof of the lemma. Suppose that $\Aa$ is a register transducer over the algebra of $k$-sourced graphs. The register transducer  $\Bb$, which is over the algebra $\aalg$, has the same input alphabet, and its registers are triples 
        \begin{align*}
        (i,j,r) \qquad \text{where $i,j \in \set{0,\ldots,k}$ and $r$ is a register of $\Aa$.}
        \end{align*}
         The register updates of $\Bb$ are defined by applying the homomorphism from the claim  to the register updates of $\Aa$. 
        The output register of $\Bb$ is $(0,0,r)$, where  $r$ is the output register of $\Aa$.  From the claim it follows 
        \begin{align*}
        \text{$\Aa$ can output $G$ on input $t$} \qquad \iff \qquad 
        \text{$\Bb$ can output $\varphi_{00}(G)$ on input $t$}.
        \end{align*}
        Since $\varphi_{00}(G)$ is the walk series of $G$ when $G$ has no sources, the lemma follows. It remains to prove the claim.

        \begin{proof} 
            It is enough to prove the claim for the case when $f$ is one of the basic operations in the algebra of $k$-sourced graphs. In particular, $Y$ has one element. Here the proof is similar to conversion of a nondeterministic automaton into a regular expression, as in the Kleene Theorem.
            
            \begin{itemize}
                \item Suppose that $f$ is a constant, which represents a  $k$-sourced graph where all vertices are sources. Then we have 
                \begin{align*}
                \varphi_{ij}(f) = \begin{cases}
                    x & \text{if there is an edge from $i$ to $j$ in $f$}\\
                    0 & \text{otherwise.}
                \end{cases}
                \end{align*}
                \item Suppose that $f$ is the  basic operation which inputs two $k$-sourced graphs and outputs their fusion.  
                An $(i,j)$-walk in the fusion $f(G_1,G_2)$ is either an $(i,j)$-walk in $G_1$ or an $(i,j)$-walk in $G_2$, this is because vertices between endpoints cannot be sources. Furthermore, these cases are disjoint, since we are working with graphs that have parallel edges.
  		This leads to the following equation for fusions:
                \begin{align*}
					\varphi_{ij}(f(G_1, G_2)) = \varphi_{ij}(G_1) + \varphi_{ij}(G_2).
                \end{align*}
                The right hand side uses component-wise addition, which is a basic operation in $\aalg$.
                \item The last case is when $f$ is a unary operation which forgets  source $c \in \set{1,\ldots,k}$.  Let then $G$ be a $k$-sourced graph. A nonempty $(i,j)$-walk in $f(G)$ is either a nonempty $(i,j)$-walk in $G$, or it has the form
                \begin{enumerate}
                    \item a nonempty $(i,c)$-walk in $G$; followed by 
                    \item finitely many (possibly zero) nonempty  $(c,c)$-walks in $G$; followed by
                    \item a nonempty $(c,i)$-walk in $G$.
                \end{enumerate}
                This leads to the following equation:
                \begin{align*}
                \varphi_{ij}(f(G)) = \varphi_{ij}(G)) +  \varphi_{ic}(G) \times (1+ \varphi_{cc}(G)^+)  \times \varphi_{cj}(G).
                \end{align*}                
            \end{itemize}
        \end{proof}
        
\end{proof}



For a rational power series, the embedding is 
\subsection{Equivalence for polynomial transducers that output power series}\label{sec:pol-transducers-any-ring}
(fill-in wstęp)
%(szkic wstępu) Equivalence of tree-to-($\Sigma^*, \cdot$) register transducers has been proved in \cite{seidlManethKemper2018}. The proof goes by simulating them with tree-to-\Z \emph{polynomial transducers}, which are a kind of register transducers that we define in next paragraph, and showing decidability of equivalence for them. To prove the main theorem we will use decidability of equivalence of tree-to-\Ring polynomial transducers, where \Ring is a ring of rational power series.

\smallparagraph{Polynomial transducers over computable rings and fields}
By a ring we always mean a \emph{commutative} ring.  Following \cite{seidlManethKemper2018}, by a \emph{polynomial transducer} we mean any tree-to-\Ring register transducer, where \Ring is some ring.

Let $\Ring$ be a ring. We call $\Ring$ \emph{computable} if its elements can be enumerated in a way that addition, multiplication (and equality test, if the enumeration is not unique) are computable functions. A \emph{nonzero} element $r$ of $\Ring$ is called a \emph{zero divisor} if $r \cdot s = 0$ for some $s \neq 0$.

We are mainly interested in rings that are subrings of some field;
for example \Z is a subring of \Q. In general, the following fact holds
\begin{lemma}\label{lem:ring-with-no-zero-divisors-subring-of-a-field}
	Let $\Ring$ be a ring. Then
	\begin{enumerate}[(i)]
		\item $\Ring$ is a subring of some field iff it has no zero divisors,
		\item if $\Ring$ is computable then the embedding can be done effectively.
	\end{enumerate}
\end{lemma}
\begin{proof}[Proof sketch]
	(Computable) ring $\Ring$ with no zero divisors is (effectively) a subring of its field of fractions.
\end{proof}


Equivalence of tree-to-$\Ring$ polynomial transducers, where $\Ring$ is the ring of integers, has been proved decidable in \cite{seidlManethKemper2018}.
In this section we observe that the same proof works for $\Ring$ being an arbitrary computable field, and what follows, for any $\Ring$ that is (effectively) a subring of a computable field (Theorem \ref{thm:equivalence-polynomial-automata-over-a-ring}). In Corollary \ref{cor:equivalence-pol-transducers-Zrat} we conclude that it is decidable for polynomial transducers that output polynomials or rational power series which we use later in Section \ref{sec:decide-power-series} and in Appendix.


\smallparagraph{The proof}
We sketch the proof of decidability of equivalence of tree-to-\Z polynomial transducers, following \cite{seidlManethKemper2018}.

\Z is a subring of field \Q and the proof is done for tree-to-\Q transducers.
The proof goes by overapproximation, for each control state $q \in Q$, the set of configurations reachable in $q$ by a set of solutions of a system of polynomial equations (i.e. Zariski closed set).
$Q$-tuple of (polynomial) ideals that represent those systems forms an \emph{inductive invariant}.
The main algorithm runs two semiprocedures in parallel -- one searching for a counterexample to equivalence, one searching for an inductive invariant that certifies equivalence. The algorithm can be formulated for any ring\footnote{And even for any algebra in fact. However, we do not discuss this further due to lack of place.}, but it works for field \Q for the following reasons.

The certificate of equivalence (inductive invariant) is a tuple of ideals that satisfy certain inclusions. Testing if those inclusions hold is done by Buchberger's algorithm. Every ideal can be finitely represented due to Hilbert's Basis Theorem and in consequence, tuples of them can be enumerated.
The proof of existence of inductive invariants (for a pair of equivalent polynomial transducers) uses \cite[Lemma 6.3]{seidlManethKemper2018}, let us denote it \emph{Product Lemma}, about ideals of polynomials that zero on cartesian products of sets.

Hilbert's Basis Theorem and Product Lemma hold for any field, with the same proof as for \Q (in fact, this is stated explicitly in \cite{seidlManethKemper2018} before Product Lemma). Buchberger's algorithm works for any computable field as well.
Therefore we obtain the following result.
\begin{theorem}\label{thm:equivalence-polynomial-automata-over-a-ring}
	Equivalence of tree-to-$\Ring$ polynomial transducers is decidable where $\Ring$ is a ring with no zero divisors.
\end{theorem}
We will use the following Corollary.
\begin{corollary}\label{cor:equivalence-pol-transducers-Zrat}
	Equivalence is decidable for tree-to-$\Z[\enum x1n]$ and tree-to-\Zrat polynomial transducers.
\end{corollary}
\begin{proof}
	This is an application of Therem \ref{thm:equivalence-polynomial-automata-over-a-ring} -- both ring of polynomial and ring of power series have no zero divisors.
\end{proof}
%\begin{remark}
%Product Lemma is used in the proof only if transducers' input alphabet contains a symbol of rank larger than 1. Therefore for \emph{string}-to-$\Ring$ transducers, Theorem generalises to rings that have finite representations of ideals and have decidable ideal inclusion -- there are examples of such rings, which are not a subring of some field.
%\end{remark}
\subsection{Equivalence for register transducers that compute walk series}\label{sec:decide-power-series}
By Lemmas~\ref{lem:transduction-to-registers} and~\ref{lem:compute-power-series}, the  equivalence problem from Theorem~\ref{thm:path-equivalence} reduces to equivalence for functional \regTsover{\aalg}, where $\aalg$ is the algebra from Lemma~\ref{lem:compute-power-series}. In this section, we complete the proof the theorem by showing that the latter problem is decidable.
\begin{lemma}\label{lem:functionality-decidable-power-series} If $\aalg$ is the algebra from Lemma~\ref{lem:compute-power-series}, then equivalence is decidable for functional \regTsover{\aalg}. 
\end{lemma}
\begin{proof}
	We will reduce to equivalence of \polTsover{\Zrat}, which is shown decidable in Corollary \ref{cor:equivalence-pol-transducers-Zrat}.
	The following equation allows to simulate Kleene plus by division operation; observe that for any $f$ we have $ff^+ + f = f^+$, and in consequence
	$$
	f^+ = \frac{f}{1-f}.
	$$
	This makes \regTover{\aalg} a \polTover{\Zrat} \emph{with division}, i.e. a polynomial transducer with additional feature that register updates can use division operation.
	The rest of the proof is devoted to showing decidability of equivalence of polynomial transducers with division.
	\begin{lemma}
		Equivalence of \polTsover{\Ring} with division is decidable, for a ring with no zero divisors $\Ring$.
	\end{lemma}
	\begin{proof}
		We will adapt the construction of projective line.
		Consider the following representation of elements of $\Ring$: a pair $(x,y)\in \Ring^2$ represents $\frac{x}{y} \in \Ring$. This representation is not unique, and is undefined on some pairs. Then addition, multiplication and, importantly, division of elements of $\Ring$ can be realised as polynomial operations on their representations in $\Ring^2$. This is employed in definition of the following algebra \balg:
		%\begin{align*}
		%	&\frac{x}{y} + \frac{z}{t} = \frac{xt+yz}{yt},\\ 
		%	&\frac{x}{y} \cdot \frac{z}{t} = \frac{xz}{yt},\\
		%	&\left(\frac{x}{y}\right)^+ = \frac{\frac{x}{y}}{1-\frac{x}{y}} = \frac{x}{y-x},
		%\end{align*}
		%and define algebra $\balg$ the following way:
		\algebradefinition{$\Ring \times \Ring$}{$+_B, \cdot_B$ of arity 2, $^+$ of arity 1:
			\begin{align*}
			&(x,y) +_B (z,t) = (xt+yz, yt),\\
			&(x,y)\cdot_B(z,t) = (xz, yt),\\
			&(x,y)^{+_B} = (x, y-x).
			\end{align*}}
		
		Also observe that there is a ''zero-testing'' polynomial function $\eqzeroinalgb$ that satisfies
		$$
		\eqzeroinalgb(x,y) = 0 \iff (x,y) \text { represents } 0,
		$$
		that is, $\eqzeroinalgb(x,y) = x$.
		
		Now, for $\Aa$ being \polTover{\Ring} with division, one can make a \polTover{\balg} $\Bb$, using equations above. Then for each input tree $t$, $\Bb(t)$ represent $\Aa(t)$.
		
		Thus the reduction is as follows: for polynomial transducers with division $\Aa_1, \Aa_2$ take polynomial transducer $\Bb_{1,2}$ corresponding to $\Aa_1-\Aa_2$. Then $\Aa_1 \equiv \Aa_2$
		%\Leftrightarrow
		if and only if 
		$\eqzeroinalgb(\Bb_{1,2}) \equiv 0$,
		that is,
		polynomial transducer $\eqzeroinalgb(\Bb_{1,2})$ obtained by composing output function of $\Bb_{1,2}$ with $\eqzeroinalgb$, is equivalent to $0$.
	\end{proof}	
\end{proof}

