\section{Graphs, treewidth, and logic on graphs}
The graphs in this paper are finite and undirected, with parallel edges, and possible self-loops. Formally, a \emph{graph} consists of: a set of \emph{vertices}, a set of \emph{edges} and a function that associates to each edge a set of at most two endpoints. We call distinct edges \emph{parallel} if their sets of endpoints are equal.
%  All theorems we cite (for example \cite[Proposition 4.1]{courcelleEtAlAlgebraicTheoryOfGraphReduction93}, \cite[Theorem 2]{groheDellRattan2018}) were originally formulated for graphs with no parallel edges; however, for the results 
\subsection{Treewidth} 
\label{sec:treewidth-definition}
As mentioned in the introduction, we work with graphs of bounded treewidth to avoid undecidability for \mso transductions. 
For treewidth, we use Courcelle's  approach via universal algebra \cite{courcelleEtAlAlgebraicTheoryOfGraphReduction93}. (Later, we also use algebra in the classical sense: rings, fields, polynomials, etc.)


\smallparagraph{Algebras and terms operations} We begin by introducing some terminology from universal algebra. By an \emph{algebra}, we mean an underlying set equipped with some  operations, e.g.~the ring of integers $(\Int, + , \times, 0, 1)$ or the free monoid $(\Sigma^*, \cdot, \varepsilon)$. We write $\aalg, \balg, \ldots$ for algebras. By abuse of notation, we write $a \in \aalg$ if $a$ belongs to the underlying set in the algebra $\aalg$. 
We use the name \emph{basic operations} for the operations that are given in the algebra. This is to distinguish them from the more general \emph{term operations}, which are defined below.
        For a finite sets of variables $X$ and an algebra $\aalg$, a function $f : \aalg^X \to \aalg,$
            is called a \emph{term operation} if it is induced by a  term  constructed using the basic operations of the algebra and variables from $X$.  If the set $X$ is $\set{0,\ldots,k-1}$ then we talk about term operations of type $\aalg^k \to \aalg$.
        For example, the polynomial $3x^2 -2x+1$ can be viewed as a term in the algebra
            $(\Real, +,\times, -1,0,1)$,
        and it induces a term operation of type $\Real \to \Real$. 
        Polynomials with non-integer coefficients such as $0.5$  are not term operations in the algebra from the above example; for this the coefficients would need to be added as constants to the algebra. We will also allow term operations that are vectorial: a function $f : \aalg^X \to \aalg^Y$ is a term operation if for every $y \in Y$, the composition of $f$ with the projection to coordinate $y$ is a term operation. 
        % In the same algebra, an example of a  for this the 
        % while the operation $x \mapsto xa$ which appends an 
        %  the operation $x \mapsto xx$ is a unary term operation. 
        % set $X$ of variables,  define a \emph{term operation with  variables $X$ in an algebra $\aalg$} to be  any operation of type $\aalg^X \to \aalg$ that arises from a term constructed using  variables $X$ and operations of the algebra.  For example, $x \mapsto xxx$ is a term operation in the free monoid. 
        % If the set of variables is $X=\set{1,\ldots,k}$, then we talk about $k$-ary operations. We  also use 
        %  \emph{vectorial term operation};  is a function
        
        % which arises from a family of term operations $f_y : \aalg^X \to \aalg$, one for each $y \in Y$.
        %  For example,  the term $xyxxy$ over variables $\set{x,y}$ in the free monoid gives rise to a binary term operation in the free monoid. 
        % A \emph{polynomial operation with variables $X$} is defined in the same way, except that the term can also use all elements of the algebra as constants.  Here is another example, which explains the terminology: if  the algebra $\aalg$ is 
        % $
        % (\Real, + , \times, 0 ,1)
        % $
        % then the $k$-ary term operations are the same as polynomials with $k$ variables with coefficients in $\set{0,1,\ldots}$, while the $k$-ary polynomial operations are the same as (general) polynomials with $k$ variables.

        
        
        \begin{wrapfigure}{r}{0.3\textwidth}
            \mypic{2}     
             \end{wrapfigure}  \smallparagraph{The algebra of $k$-sourced graphs}
To define treewidth, we consider graphs with distinguished vertices and equip them with an algebraic structure.  
    For $k \in \set{1,2,\ldots}$,  define a \emph{$k$-sourced graph}
     to be a  graph together with a $k$-tuple of pairwise distinct vertices called \emph{sources}. We allow sources to be undefined, for example the 3-sourced graph in the picture to the right has source 2 undefined.
    We view  $k$-sourced graphs as an algebra, in the following sense.



    
    \begin{definition}
        [Algebra of $k$-sourced graphs] Let $k \in \set{1,2,\ldots}$. The \emph{algebra of $k$-sourced graphs} has as its underlying set the  $k$-sourced graphs,  with the following basic operations:
            \begin{itemize}
                \item {\bf Constants.} There is a constant for every $k$-sourced graph where all vertices are sources and there is at most one edge.
                \item {\bf Forget.} For every $i \in \set{1,\ldots,k}$ there is a unary operation, which inputs a $k$-sourced graph and outputs the same $k$-sourced graph, except that the $i$-th source is undefined. 
                \item {\bf Fuse.} There is one binary operation, called \emph{fusion}, which inputs two $k$-sourced graphs, and 
                connects %ale join też bardzo pasuje
                them along same-numbered sources, 
                %outputs their disjoint union, with same-numbered sources being identified, 
                as explained in the following picture (note that fusion of graphs with no parallel edges might contain parallel edges):
                \mypic{1}
            \end{itemize}
        \end{definition}
    
        Terms in this algebra correspond to tree decompositions in the original sense of Robertson and Seymour~\cite[Section 12.3]{diestel}. More precisely, a graph has treewidth $\leq k$  if and only if it can be produced using the operations of the algebra of $(k+1)$-sourced graphs \cite[Proposition 4.1]{courcelleEtAlAlgebraicTheoryOfGraphReduction93}. This justifies the following definition.


  \begin{definition}[Treewidth]
        For $k \in \set{1,2,\ldots}$, we say that a  graph has treewidth  $\le  k$ if it is the value of some term in the algebra of $(k+1)$-sourced graphs.
    \end{definition}

  
    
    % The definition above coincides with the original definition of Robertson and Seymour, which uses tree decompositions with bags of vertices~\cite[Proposition 1.19]{courcelle1991}. 

    \subsection{Monadic second-order logic}
    We  use logic, especially monadic second-order logic, to define properties and transformations of graphs. 

    \smallparagraph{Transductions}
    A  \emph{vocabulary} is defined to be  a set of relation names, each one with an associated arity. A \emph{model over vocabulary $\tau$} consists of a set, called the \emph{universe} of the model, together with an \emph{interpretation}, that maps the relation names from the vocabulary $\tau$ to  relations over the universe of same arity. To define properties of such models, we use mainly monadic second-order logic \mso, which  is the extension of first-order logic that allows quantification over sets of elements (but not sets of pairs, or triples, etc.). We assume that the reader is familiar with \mso, for an introduction see~\cite[Section 2]{Thomas97}.
     
     
 The central topic of this paper is \mso transductions, which are graph transformations definable in \mso (from now on, when talking about \mso we mean graph properties that are definable in monadic second-order logic using the \msotwo representation of graphs as models). The transductions can be nondeterministic, in which case they represent binary relations and not functions.
A \emph{transduction}, with input vocabulary $\tau$ and output vocabulary $\sigma$ is defined to be a set of pairs 
\begin{align*}
    (\myunderbrace{\text{model over $\tau$}}{input model}, \ 
    \myunderbrace{\text{model over $\sigma$}}{output model}),
\end{align*}
which is isomorphism invariant, in the sense that membership in the transduction is not affected by changing a model -- on either the input or output coordinate -- to an isomorphic one. The central topic of this paper is transductions that can be defined using \mso, as described in the following definition, whose phrasing is based on~\cite[p.~9--10]{bojanczykOptimizingTreeDecompositions2017a}. 

\begin{definition}[\mso transduction]\label{def:mso-transduction}
    An \mso transduction is any transduction that can be obtained by composing a finite number of transductions of the following kinds.
Kind 1 is a partial function, kinds 2, 3, 4 are functions, and kind 5 is a relation.
\begin{enumerate}
	\item {\bf Filtering.} \label{mso-trans:filtering} For every \mso sentence $\varphi$ over the input vocabulary there is a transduction that filters out models where $\varphi$ is violated. Formally, the transduction is the partial identity -- with input and output vocabularies being equal --  whose domain consists of the models that satisfy the sentence. 
	\item {\bf Universe restriction.} For every \mso formula $\varphi(x)$ over the input vocabulary with one free first-order variable, there is a transduction, which restricts the universe of the input model to those elements that satisfy $\varphi$. The input and output vocabularies are the same, the interpretation of each relation in the output model is defined as the restriction of its interpretation 
	in the input model to tuples of elements that remain in the universe.
	\item {\bfseries{\scshape{Mso}} interpretation.} This kind of transduction changes the vocabulary of the model while keeping the universe intact. To specify an \mso interpretation, for for every relation name $R$ of the output vocabulary, one gives an \mso formula $\varphi_R(x_1,\ldots,x_k)$ over the input vocabulary which has as many free first-order variables as the arity of $R$. The output model is obtained from the input model by keeping the same universe, and interpreting each relation $R$ of the output vocabulary as the set of tuples  satisfying $\varphi_R$.
	\item {\bf Copying.} \label{mso-trans:copying} For  $k \in \set{1,2,\ldots}$, define $k$-copying to be the transduction which inputs a model and outputs a model consisting of $k$ disjoint copies of the input, with an additional relation that ties the copies together. Formally, the output universe consists of $k$ copies of the input universe.
	The output vocabulary is the input vocabulary enriched with a binary predicate $\mathsf{copy}$ which selects pairs which originate from the same element, and unary predicates $\mathsf{layer}_1,\mathsf{layer}_2,\ldots,\mathsf{layer}_k$ which select elements belonging to the first, second, etc. copies of the universe.
	In the output model, a relation name $R$ of the input vocabulary is interpreted as the set of all those tuples over the output model, which are in the same layer, and  where the original elements of the copies were in relation $R$
	in the input model.
	\item {\bf Colouring.} \label{mso-trans:colouring} Colouring adds  a new unary predicate to the input model, and interprets it nondeterministically. Formally, the universe as well as the interpretations of all relation names of the input vocabulary stay intact,
	but the output vocabulary has one more unary predicate. For every possible interpretation of this unary predicate, there is a different output with this interpretation implemented.
\end{enumerate}
\end{definition}

\smallparagraph{Logic and transductions on graphs}
To define properties and transductions for graphs in \mso, we represent graphs as models. There are at least two ways to do this, called  \msoone and \msotwo following 
Courcelle~\cite[Definition 1.7]{courcelleVI}. In this paper, we use the more expressive \msotwo  model: the universe is the disjoint union of vertices and edges, and there is a binary  \emph{incidence relation} that consists of pairs $(v,e)$ such that vertex $v$ participates in edge $e$.  We present our results for monadic second-order logic without modulo counting, but all of our results can be easily extended to include modulo counting. 

% In this paper, we use the \msotwo model. For general graphs, with or without parallel edges, more properties of graphs can be expressed in \mso using the \msotwo representation, e.g.~existence of a Hamiltonian cycle can be expressed using the \msotwo representation but not with the \msoone representation~\cite[p.~118]{courcelleVI}. However, for graphs of bounded treewidth that have no parallel edges, \msoone and \msotwo have the same expressive power~\cite[Theorem 9.37]{courcelleGraphStructureMonadic2012}.
Define a \emph{graph-to-graph \mso transduction} to be an \mso transduction, where the input and output vocabularies are the same, namely the vocabulary of graphs (one binary relation for incidence), and which only contains pairs of graphs (more formally, pairs of \msotwo models of graphs).  By abuse of notation, when $\Tt$ is a graph-to-graph \mso transduction, and $G,H$ are graphs, we write $(G,H) \in \Tt$ instead of saying that $\Tt$ contains the  pair (\msotwo model of $G$, \msotwo model of $H$).
If $\Tt$ is functional, which means that for every input graph there is at most one output graph up to isomorphism, then we write $\Tt(G)$ for the (isomorphism type of the) resulting graph (which may be undefined, because $\Tt$ might be a partial function). 

\begin{example}
    Consider the partial function which maps a clique of size $n^2$ to an $n\times n$ grid, and is undefined for other inputs.  This is an \mso transduction: (1) nondeterministically colour the edges to select those that participate in the grid; (2) remove the other edges; (3) check that the result is indeed a grid. For this example it is important that we use the \msotwo representation, which enables colouring edges.  Also note  different sets of edges might be removed in steps (2), but the result is always the same up to isomorphism.
\end{example}    

\smallparagraph{The equivalence problem}
The topic of this paper is   the following  decision problem for functional \mso graph-to-graph transductions. 

\decisionproblem{equivalence for functional graph-to-graph \mso transductions of bounded treewidth}{ $\ell, k \in \set{1,2,\ldots}$ and two functional graph-to-graph \mso transductions $\Tt_1,\Tt_2$.}{is it the case that for every input $G$ of treewidth at most $\ell$, the two outputs  $\Tt_1(G),\Tt_2(G)$ are isomorphic and have treewidth at most $k$?}

If the  transductions in the instance are not functional, then there are no requirements on the answer to the algorithm. This motivates the following problem:

\decisionproblem{functionality for graph-to-graph \mso transdcutions of bounded treewidth.}{ $\ell, k \in \set{1,2,\ldots}$ and a graph-to-graph \mso transduction,  not necessarily functional.}{is it the case that for every input  of treewidth at most $\ell$, there is at most one output, and this output -- if defined -- has  treewidth at most $k$?}
    
We do not know how to solve any of the above problems, but at least we know that they are equi-decidable, i.e.~if one is decidable then the other is decidable. 

\begin{fact}\label{fact:equi-decidable}
    The equivalence and functionality problems described above are equi-decidable.
\end{fact}
\begin{proof}
In both problems we require that: (*) if the output has treewidth at most $\ell$, then every output has treewidth at most $k$. This is a decidable property, because the graphs of treewidth strictly bigger than  $k$ are definable in \mso (as a complement of  a minor closed class),  \mso definable languages are closed under inverse images of \mso transductions~\cite[Backwards Translation Theorem]{courcelleGraphStructureMonadic2012}, and satisfiability of \mso is decidable on graphs of treewidth at most $\ell$ \cite[Theorem 5.80]{courcelleGraphStructureMonadic2012}. 

We can therefore assume that the transductions at hand satisfy (*). 
    Equivalence reduces to functionality, because two transductions are  equivalent if and only if they have the same domains when restricted to treewidth at most $\ell$, and their union is  functional. The domain equivalence can be checked in the same way as (*).

    Consider now the converse reduction: from functionality to equivalence. Suppose that $\Tt$ is an \mso transduction, and we want to check if $\Tt$ is functional for inputs of treewidth at most $\ell$ and outputs of treewidth at most $k$. Because colourings can be pushed to the head of an \mso transduction, see~\cite[Section 1.7]{courcelleGraphStructureMonadic2012}, we can decompose $\Tt$ as $\mathcal T =  \mathcal F \circ \mathcal S$, where:
    \begin{itemize}
        \item $\mathcal C$ is a colouring, i.e.~a composition of transductions as in item~\ref{mso-trans:colouring} of Definition~\ref{def:mso-transduction}.
        \item $\mathcal F$ is a composition of transductions from items~\ref{mso-trans:filtering}-\ref{mso-trans:copying} of Definition~\ref{def:mso-transduction}.
    \end{itemize}
    In particular, since items~\ref{mso-trans:filtering}-\ref{mso-trans:copying} of Definition~\ref{def:mso-transduction} describe partial functions,  $\mathcal F$ is a partial function. There is some finite set of colours $C$ (if $\Ff$ is a composition of $n$ copying transductions, then $C=2^n$) such that the  transduction $\mathcal C$ inputs a graph, and nondeterministically outputs a $C$-coloured graph (i.e.~a graph with vertices labelled by colours from $C$). For $i  \in \set{1,2}$, define $\Ff_i$ to be the \mso transduction which inputs a $(C \times C)$-coloured  graph, projects the colours to the $i$-th coordinate, and then applies $\Ff$. The transduction $\Tt$ is functional if and only if the transductions $\Ff_1$ and $\Ff_2$ are equivalent (and the same statement is true if we restrict inputs and outputs to have given treewidth). Therefore, the functionality problem reduces to the equivalence problem for coloured graphs. Finally, colours can be simulated by the graph structure without affecting treewidth,  as explained in the following picture:
    \mypic{5}
    Let $\Tt$ be the \mso transduction from coloured graphs to uncoloured graphs, and let $\Tt^{-1}$ be its one-sided inverse, which is a partial function from uncoloured graphs to coloured graphs.
    % \begin{align*}
    %     \xymatrix@C=3cm{
    %         \text{$C$-coloured graphs}
    %         \ar@/^1pc/[r]^-{\Tt} &
    %         \text{graphs}.
    %         \ar@/^1pc/[l]^-{\text{partial function }\Tt^{-1}}
    %     }
    %     \end{align*}
    Both $\Tt$ and $\Tt^{-1}$ do not change treewidth. 
    Two functional \mso transductions $\Ff_1$ and $\Ff_2$  from coloured graphs to coloured graphs are equivalent if and only if the functional graph-to-graph \mso transductions  $\Tt \circ \Ff_1 \circ \Tt^{-1}$ and $\Tt \circ \Ff_2 \circ \Tt^{-1}$ are equivalent. 
     Therefore, the functionality and equivalence problems for coloured graphs reduce to the same problems for un-coloured graphs.
    % (constructing $T_1 \vee T_2$ can be done e.g. by adding a coordinate to colouring that tells which one of $T_1$, $T_2$ is used). For the opposite direction, functionality can be reduced to equivalence of transductions where input is labelled: for a given transduction $T$, test equivalence of two (deterministic) transductions that input graphs labelled with pairs of colours used by $T$ and execute $T$, using first, respectively second colouring. (fill-in (resolve labelled input))
    % For every $k \in \set{1,2\ldots}$, the set of graphs of treewidth $k$ is \mso definable.  (A quick justification is that the graphs of treewidth $> k$ are characterised by forbidden minors~\cite{}, and having a fixed graph as a minor is an \mso definable property.) Therefore, thanks to the filtering step in \mso transductions, we can assume that the the two \mso transductions in the instance of the above decision problem have only outputs of treewidth at most $\ell$, and have only outputs of treewidth at most $k$. 
\end{proof}

The rest of this paper is devoted to a discussion of the above two problems, including a solution for a variant where output graphs are identified up to a relaxation of isomorphism.
%As we explain below, we do not even know how to solve the problem for $k=\ell=1$, i.e.~for \mso transductions which input and output forests.
We begin with two examples which illustrate the fact that equivalence problem for graph-to-graph transductions is unsolved even for very restricted instances.

\noindent{\bf Example: treewidth 1.} 
    Consider the
    %special
    case
    %of the equivalence problem for functional graph-to-graph \mso transductions of bounded treewidth, 
    where both input and output graphs have treewidth 1. (As we will see in Lemma~\ref{lem:transduction-to-registers}, it is the output treewidth that is important, since the input treewidth can always be reduced to 1 without making the problem any easier.) This is a very special case of the equivalence problem, but even this special case we do not know how to solve.
    
    Let us compare this 
    %special case 
    with a very similar problem that is already known to be decidable. There are three kinds of trees that can be considered, listed in order of increasing rigidity: (a) without a distinguished root and without sibling order; (b)  with a distinguished root and without sibling order; (c) with a distinguished root and with sibling order. The (a) variant is the special case of equivalence for graphs of treewidth 1.  As discussed in the introduction, the equivalence problem for \mso transductions is harder in  variant (b) than in variant (c), and for similar reasons (a) is harder than (b). Variant (b) is known to be decidable, but variant (a) remains open.  One idea would be to reduce variant (a) to variant (b) by transforming an  un-rooted tree into the rooted forest of all possible ways of choosing a root; this transformation has quadratic size outputs, and therefore it cannot be an \mso transduction.
    

    \noindent{\bf Example: strings modulo cyclic shift.} Here is another special case of our problem, which seems entertaining and also is open.
    % \begin{example}
        Define \emph{cyclic equivalence} to be the equivalence relation $\sim$ on strings which identifies two strings modulo cyclic shifts (for example $abcd \sim cdab$).
         \begin{wrapfigure}{r}{0.5\textwidth}
            \mypic{9}
         \end{wrapfigure}
         Consider the following  problem: given two functional string-to-string \mso transductions~\cite[Definition 1]{engelfrietMSODefinableString2001},
        decide if for every input string, the two output strings are equal  modulo cyclic equivalence. %	This seems to be an entertaining problem.
         It is a special case of the graph-to-graph equivalence problem, since  strings modulo cyclic equivalence  can be represented as graphs of treewidth at most 2, as explained in the  picture.
        % \end{example}


        