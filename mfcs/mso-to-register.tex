\section{From transductions to register transducers}
\label{sec:transductions-to-transducers}
This part of the appendix proves Lemma~\ref{lem:transduction-to-registers}. The statement of the lemma is as follows. 
    Let $\ell, k \in \set{1,2,\ldots}$, and  let $\Tt$ be an \mso transduction where all output graphs have treewidth at most $k$. There is a nondeterministic \treetotreewidth{ k} transducer $\Aa$ which makes the following diagram commute (arrows in the diagram are binary relations):
    \begin{align*}
    \xymatrix@C=0.3cm{
         & \txt{tree decompositions of width $< \ell$}
        \ar[dl]_-{\txt{\scriptsize underlying graph \qquad \qquad}}
         \ar[dr]^\Aa\\
        \txt{graphs of treewidth $< \ell$}\ar[rr]_\Tt & &
        \txt{graphs  of treewidth $< k$} 
    }
    \end{align*}


Defined an \emph{unranked tree decomposition of width $k$} in the same way as a tree decomposition of width $k$, except that the fusion operation is now viewed as unranked (i.e.~it can have any finite number of children, which are unordered). An unranked tree decomposition can be viewed as a model, where the universe is the nodes, there is binary child relation, and for each  node label (which is a name of an operation in the algebra of $k$-sourced graphs) there is a unary relation that selects nodes with that label.  In~\cite[Corollary 3]{bojanczykOptimizingTreeDecompositions2017a}, it is shown that \mso transductions can compute (nondeterministically) tree decompositions for graphs of bounded treewidth, as stated in the following theorem.
 
\begin{theorem}\label{thm:boj-pil}
    For every $k \in \set{1,2,\ldots}$ there is an \mso transduction $\Ss$ which makes the following diagram commute (arrows are relations, the double arrow is the identity):
    \begin{align*}
        \xymatrix@C=-0.3cm{ 
              & \txt{graphs of treewidth $<k$}
             \ar[dl]_{\Ss}
             \ar@{-}@<-.5ex>[dr] \ar@{-}@<.5ex>[dr]
             \\
            \txt{unranked\\ tree decompositions\\
            of width $\le k$} 
            \ar[rr]_-{\text{ \qquad underlying graph}}
             & & \txt{graphs of treewidth $<k$}
        }
    \end{align*}
\end{theorem}

Apply Theorem~\ref{thm:boj-pil}, yielding an \mso transduction $\Ss$.
Define $\Rr$ to  be the composition of the  red arrows in the following diagram:
\begin{align*}
    \xymatrix@C=3cm{
        \txt{tree decompositions\\ of width $\le \ell$}
        \ar@[red][d]_{\txt{\scriptsize underlying graph}}
         \ar[r]^\Rr & 
         \txt{unranked \\ 
         tree decompositions\\
         of width $\le k$}
         \\
        \txt{graphs of\\ treewidth $\le \ell$}\ar@[red][r]_\Tt &
        \txt{graphs  of\\ treewidth $\le k$} 
        \ar@[red][u]_{\Ss}
    }
    \end{align*}
Like every composition of \mso transductions, $\Rr$ is an \mso transduction. Therefore, it is computed by a nondeterministic tree-to-forest-algebra register transducer.