\section{Perspectives}
In this section we discuss future perspectives of our approach. First direction is taking other polynomials or series. Let us start with two well known graph polynomials: Tutte polynomial and characteristic polynomial.

Tutte polynomial, denote it $T(G)$, counts many substructures of given "rank" in $G$, e.g. spanning forests, colourings. If we assume that a register transducer $\join$s along at most one vertex at a time, then, due to recurrences $T(G \cup H) = T(G) \cdot T(H), T(G\sqcup_a H) = T(G) \cdot T(H)$, Tutte polynomial of outputted graph can be easily computed by a register transducer. The distinguishing power of Tutte polynomial is however very low, at least in our context: all trees with $n$ edges have the same Tutte polynomial ($x^n$).

Let us now move to characteristic polynomial. Interestingly, two graphs have the same characteristic polynomial if and only if they have the same numbers of length-$n$ closed walks for all $n$ (cf. remarks after Theorem \ref{thm:grohe}). Let us assume again that a register transducer $\join$s only along at most one vertex at a time. For a sourced graph $G$, define $\varphi(G) = ({\widetilde{G}}^a, \overline{G}^a)_{a \in \sources(G)}$, where $({\widetilde{G}}^a, \overline{G}^a)_{a \in \sources(G)}$ is the following "split" of characteristic polynomial, induced by Laplace expansion of $A_G-t\cdot I$ at a row corresponding to vertex $a$:
$$
	\det(A_G-t\cdot I) = (-t)\cdot \underbrace{\det(A_{G\setminus a} - t\cdot I)}_{{\widetilde{G}}^a} + \underbrace{\sum_{v \in (G-a)}G^a_v \cdot \ldots}_{\overline{G}^a}.
$$
Now take two graphs $G, H$ with a common vertex $a$. If we order vertices of $G$ and $H$ so that $a$ is the last vertex of $G$ and the first vertex of $H$, adjacency matrix of $G\sqcup_aH$ looks like this:\\
%\documentclass{article}
%\usepackage{tikz}
%\usepackage{geometry}
%\usepackage{amsmath}
%\usetikzlibrary{matrix,positioning}

%\begin{document}
\begin{tikzpicture}[
%style
stylmacierzy/.style={
	matrix of math nodes,
	every node/.append style={text width=\sizemacierzy, text height=\sizemacierzy,align=center},
	nodes in empty cells,
	left delimiter=[,
	right delimiter=],
}
]
\matrix[stylmacierzy] (matadjacency)
{
	& & & & & & & & \\
	& & & & & & & & \\
	& & & & & & & & \\
	& & & & & & & & \\
	& & & & & & & & \\
	& & & & & & & & \\
	& & & & & & & & \\
	& & & & & & & & \\
	& & & & & & & & \\
};
\draw[]%pionowy prawy po G'
	(matadjacency-1-3.north east) -- (matadjacency-3-3.south east);
\draw[]%poziomy dolny po G'
	(matadjacency-3-1.south west) -- (matadjacency-3-3.south east);
\draw[]%pionowy lewy boczny po H'
	(matadjacency-5-5.north west) -- (matadjacency-5-9.north east);
\draw[]%górny po H'
	(matadjacency-5-5.north west) -- (matadjacency-9-5.south west);	
\draw[loosely dashed]%% poziomy przy Ga
	(matadjacency-3-4.south west) -- (matadjacency-3-9.south east);
\draw[loosely dashed]%poziomy górny przy Ha
	(matadjacency-4-1.south west) -- (matadjacency-4-4.south east);
\draw[loosely dashed]%lewe dolne 0, prawy pionowy
	(matadjacency-4-3.north east) -- (matadjacency-9-3.south east);	
\draw[loosely dashed]%prawe górne 0, lewy pionowy
	(matadjacency-1-5.north west) -- (matadjacency-4-5.south west);	
%
\node[font=\GprimSize] 
	at (matadjacency-2-2) {$G'$};
\node[font=\GaSize] 
	at (matadjacency-2-4) {$G^a$};
\node[font=\GaSize] 
	at (matadjacency-4-2) {$G^a$};
\node[font=\Large] 
	at (matadjacency-4-4) {$a$};
\node[font=\GaSize] 
at (matadjacency-4-7) {$H^a$};
\node[font=\GaSize] 
at (matadjacency-7-4) {$H^a$};
\node[font=\huge]
	at (matadjacency-7-7) {$H'$};
\node[font=\ZeroSize]%prawe górne 0
	at (matadjacency-2-7) {$0$};
\node[font=\ZeroSize]
	at (matadjacency-7-2) {$0$};
%
%\draw[]
%	(matadjacency-3-1.south) --
%	node[left=41pt] {Laplace}
%	(matadjacency-3-1.south)
%	;
% \draw[]
%	(matadjacency-4-1) --
%	node[left=25pt] {expansion - - -}
%	(matadjacency-4-1)
%	;
\end{tikzpicture}

%\end{document}
Now, take Laplace expansion at row corresponding to vertex $a$:
\begin{multline*}
	\det(A_{G\sqcup_a H}) = \sum_{v \in (G- a)} G^1_v \cdot \ldots + (-t) \cdot \ldots + \sum_{w \in (H- a)} H^1_w \cdot \ldots = \ldots (\text{ to be ctd. })
\end{multline*}
Observe that $v$-cofactor, $a$-cofactor and $w$-cofactor, for $v\in (G-a), w \in (H-a)$ are determinants of, respectively:\\
\begin{tikzpicture}[
%style
stylmacierzy/.style={
	matrix of math nodes,
	every node/.append style={text width=\sizemacierzy, text height=\sizemacierzy,align=center},
	nodes in empty cells,
	left delimiter=[,
	right delimiter=],
}
]%to ma juz \begin{tikzpicture}[styl]
\begin{scope}
%\documentclass{article}
%\usepackage{tikz}
%\usepackage{geometry}
%\usepackage{amsmath}
%\usetikzlibrary{matrix,positioning}

%\begin{document}

	\matrix[stylmacierzy] (matadjacency)
	{
		& & & & & & & \\
		& & & & & & & \\
		& & & & & & & \\
		& & & & & & & \\
		& & & & & & & \\
		& & & & & & & \\
		& & & & & & & \\
		& & & & & & & \\
	};
	\draw[loosely dashed]%pionowy prawy po G'
		(matadjacency-1-2.north east) -- (matadjacency-3-2.south east);
	\draw[]%dolny po G'
		(matadjacency-3-1.south west) -- (matadjacency-3-2.south east);
	\draw[]%poziomy górny po H'
		(matadjacency-3-4.south west) -- (matadjacency-3-8.south east);
	\draw[]%pionowy lewy boczny po H'
		(matadjacency-4-4.north west) -- (matadjacency-8-4.south west);	
	\draw[]% poziomy przy Ga
		(matadjacency-3-3.south west) -- (matadjacency-3-3.south east);
	\draw[loosely dashed]%lewe dolne 0, prawy pionowy
		(matadjacency-4-2.north east) -- (matadjacency-8-2.south east);	
	\draw[]%prawe górne 0, lewy pionowy
		(matadjacency-1-4.north west) -- (matadjacency-3-4.south west);	
	%
	\node[font=\GprimSize] 
		at (matadjacency-2-2) {$G'_{\hat{v}}$};
	\node[font=\GaSize] 
		at (matadjacency-2-3) {$G^a$};
	\node[font=\GaSize] 
	at (matadjacency-6-3) {$H^a$};
	\node[font=\HprimSize]
		at (matadjacency-6-6) {$H'$};
	\node[font=\ZeroSize]%prawe górne 0
		at (matadjacency-2-6) {$0$};
	\node[font=\ZeroSize]
		at (matadjacency-6-2) {$0$};
	%
	%\draw[]
	%	(matadjacency-3-1.south) --
	%	node[left=41pt] {Laplace}
	%	(matadjacency-3-1.south)
	%	;
	%\draw[]
	%	(matadjacency-4-1) --
	%	node[left=25pt] {v in $G-a$}
	%	(matadjacency-4-1)
	%	;
\end{scope}
%\end{tikzpicture}

%\end{document}
\begin{scope}[xshift=\picPrzesuniecie]
%\documentclass{article}
%\usepackage{tikz}
%\usepackage{geometry}
%\usepackage{amsmath}
%\usetikzlibrary{matrix,positioning}

%\begin{document}
\begin{tikzpicture}[
%style
stylmacierzy/.style={
	matrix of math nodes,
	every node/.append style={text width=\sizemacierzy, text height=\sizemacierzy,align=center},
	nodes in empty cells,
	left delimiter=[,
	right delimiter=],
}
]
	\matrix[stylmacierzy] (matadjacency)
	{
		& & & & & & & \\
		& & & & & & & \\
		& & & & & & & \\
		& & & & & & & \\
		& & & & & & & \\
		& & & & & & & \\
		& & & & & & & \\
		& & & & & & & \\
	};
	\draw[]%pionowy prawy po G'
		(matadjacency-1-3.north east) -- (matadjacency-3-3.south east);
	\draw[]%dolny po G'
		(matadjacency-3-1.south west) -- (matadjacency-3-3.south east);
	\draw[]%poziomy górny po H'
		(matadjacency-3-4.south west) -- (matadjacency-3-8.south east);
	\draw[]%pionowy lewy boczny po H'
		(matadjacency-4-3.north east) -- (matadjacency-8-3.south east);	
	%
	\node[font=\GprimSize] 
		at (matadjacency-2-2) {$G'$};
	\node[font=\HprimSize]
		at (matadjacency-6-6) {$H'$};
	\node[font=\ZeroSize]%prawe górne 0
		at (matadjacency-2-6) {$0$};
	\node[font=\ZeroSize]%lewe dolne 0
		at (matadjacency-6-2) {$0$};
	%
	%\draw[]
	%	(matadjacency-3-1.south) --
	%	node[left=41pt] {Laplace}
	%	(matadjacency-3-1.south)
	%	;
	%\draw[]
	%	(matadjacency-4-1) --
	%	node[left=25pt] {a}
	%	(matadjacency-4-1)
	%	;

\end{tikzpicture}

%\end{document}
\begin{scope}[xshift=\picPrzesuniecie + \picPrzesuniecie]
%\documentclass{article}
%\usepackage{tikz}
%\usepackage{geometry}
%\usepackage{amsmath}
%\usetikzlibrary{matrix,positioning}

%\begin{document}

	\matrix[stylmacierzy] (matadjacency)
	{
		& & & & & & & \\
		& & & & & & & \\
		& & & & & & & \\
		& & & & & & & \\
		& & & & & & & \\
		& & & & & & & \\
		& & & & & & & \\
		& & & & & & & \\
	};
	\draw[]%pionowy prawy po G'
		(matadjacency-1-3.north east) -- (matadjacency-3-3.south east);
	\draw[]%poziomy dolny po G'
		(matadjacency-3-1.south west) -- (matadjacency-3-3.south east);
	\draw[]%poziomy górny po H'
		(matadjacency-3-4.south west) -- (matadjacency-3-8.south east);
	\draw[loosely dashed]%pionowy lewy boczny po H'
		(matadjacency-4-5.north west) -- (matadjacency-8-5.south west);	
	%\draw[]% poziomy przy Ga
	%	(matadjacency-3-4.south west) -- (matadjacency-3-4.south east);
	\draw[]%pionowy prawy przy lewe dolne 0, 
		(matadjacency-4-3.north east) -- (matadjacency-8-3.south east);	
	\draw[loosely dashed]%pionowy lewy przy prawe górne 0, 
		(matadjacency-1-5.north west) -- (matadjacency-3-5.south west);	
	%
	\node[font=\GprimSize] 
		at (matadjacency-2-2) {$G'$};
	\node[font=\GaSize] 
		at (matadjacency-2-4) {$G^a$};
	\node[font=\GaSize] 
	at (matadjacency-6-4) {$H^a$};
	\node[font=\HprimSize]
		at (matadjacency-6-6) {$H'_{\hat{w}}$};
	\node[font=\ZeroSize, xshift=-4pt]%prawe górne 0
		at (matadjacency-2-7) {$0$};
	\node[font=\ZeroSize]%lewe dolne 0
		at (matadjacency-6-2) {$0$};
	%
	%\draw[]
	%	(matadjacency-3-1.south) --
	%	node[left=41pt] {Laplace}
	%	(matadjacency-3-1.south)
	%	;
	%\draw[]
	%	(matadjacency-4-1) --
	%	node[left=25pt] {w in $H-a$}
	%	(matadjacency-4-1)
	%	;
\end{scope}
%\end{tikzpicture}

%\end{document}
\end{tikzpicture}
\\
Now, by the fact that $\left|\begin{array}{c|c}
A & \ast\\
\hline
0 & B
\end{array}\right|
=
\left|\begin{array}{c|c}
A & 0 \\
\hline
\ast & B
\end{array}\right|
=
\det(A) \cdot \det(B),
$
where $A,B$ are square matrices, possibly of different size,
we get $\det(A_{G\sqcup_a H})$ is equal to
$$
\ldots = {\widetilde{G}}^a \cdot \det(H') + (-t) \cdot \det(G') \cdot \det(H') + \det(G') \cdot {\widetilde{H}}^a,
$$
that is,
\begin{equation}
	{\widetilde{G}}^a\cdot\overline{H}^a + (-t) \cdot {\widetilde{G}}^a \cdot {\widetilde{H}}^a + \overline{G}^a \cdot {\widetilde{H}}^a. \text{}(why left aligned?)
\end{equation}
Lat us remark that \cite{contrerasGluingLaplacians20} provides the same proof of above fact. It also studies the behaviour of graph Laplacians after $\join$ operation (called \emph{interface gluing} in the paper), which are determinants of other matrices associated to graphs.
The distinguishing power of characteristic polynomial is also low in our context -- a result \cite{schwenkCospectral73} due to Schwenk says that almost all trees, in a sense of probability, are not uniquely characterized by their characteristic polynomial.

Second direction of further research might be to search for classess of graphs that admit polynomials or power series with high distinguighing power. We conjecture that (edge-labelled version of) walk series (i.e. $\sum_{n,m=0}^{\infty}\eta_{n,m} a^n b^m \in \mathbb{Z}[[a,b]]$, where $\eta_{n,m}$ equals number of walks with $n$ $a$-edges and $m$ $b$-edges) is injective on the class of edge-labelled cycles (which, by a suitable encoding, can be seen as a subclass of unlabelled graphs of pathwidth 4 and treewidth 2). %dodając 1 lub 2widełki w każdym wierzchołku cyklu, w zależności czy ma etykiete a czy b
%Jedyna literatura warta uwagi, którą udało mi się znaleźć to "Negami, Polynomial Invariants of Graphs" oraz "Polynomial invariants of graphs II Seiya Negami & Katsuhiro Ota " przeczytam je i zobaczę co da się z nich napisać.