
\section{Perspectives}
We give remarks about possible directions of further research. Before that we introduce necessary definitions.

Let $R$ be either a ring of polynomials or algebra \aalg from Lemma \ref{lem:compute-power-series} and $m, k$ any positive integers. We call a mapping $\varphi$ from $k$-sourced graphs algebra to $R^m$ \emph{compositional}, if for all graphs $G, H$ both $\varphi(\join(G, H))$ and $\varphi(\forget_i(G))$ can be expressed as a term operation of $\varphi(G), \varphi(H)$; we call it \emph{compositional with respect to $\join$/$\forget$} if the claim holds for $\join$/$\forget$ operation. We call a graph polynomial (possibly infinite) $p(G)$ \emph{compositional}, if there exists a compositional mapping $\varphi$ of sourced graphs such that $p(G)$ is $i$-th coordinate of $\varphi(G)$, for all $G$, for some $1\leq i\leq m$. A pair consisting of walk series and mapping $\varphi$ from Lemma \ref{lem:compute-power-series} is an example of such situation.

Let \Cc be a class of graphs. A graph polynomial $p(G)$ is called a complete isomorphism invariant on \Cc if it uniquely characterizes every graph from \Cc {} among graphs from \Cc, i.e. for $G,H\in \Cc$, $p(G)$ and $p(H)$ are equal if and only if $G$ and $H$ are isomorphic. %w literaturze występuje też podobna, ale inna własność: każdy graf z klasy ma unikatowy wielomian wśród *wszystkich grafów*.
As existence of a graph polynomial that is a compositional complete isomorphism invariant on \Cc would imply decidability of equivalence of bounded treewidth graph-to-\Cc {} \mso transductions (via proof analogous to one of Theorem \ref{thm:path-equivalence}), for any graph polynomial $p(G)$ we are interested if (1) it is compositional and (2) it is a complete isomorphism invariant. Question (2) can be relaxed to (2') is relation $G \sim H \iffdef p(G) = p(H)$ ''close'' to isomorphism. Both (2) and (2') can be relativised to any class \Cc.

In this paper, we answer (1) and (2') for $p(G)$ being walk series. A possible further direction is to try to do the same for already existing, well-studied graph polynomials. We will sketch such attempt in next subsection. 
\subsection{Two polynomials:Tutte and characteristic}
In this subsection we will try to answer (1) and (2') for two polynomials: Tutte polynomial and characteristic polynomial. We will show that they are compositional (with some restrictions), but unfortunately they are very far from being a complete isomorphism invariant on the class of graphs of treewidth 1. This does not imply though that they are not ''close'' on classess of larger treewidth. The definitions of Tutte and characteristic polynomial and proofs of following claims are transferred to Appendix section; in fact, Lemma \ref{lem:Tutte-copositional} and Fact \ref{fact:Tutte-distinguishing-power} are immediate consequences of well-known formulas, and for Fact \ref{fact:characteristic-distinguishing-power} we provide reference; only the proof of Lemma \ref{lem:characteristic-compositional} is new.

Let $G$ be a graph. Denote its Tutte polynomial by $T(G)$ and its characteristic polynomial by $\charact(G)$. Let us recall that polynomial functions are term operations.
\begin{restatable}{lemmarestat}{lemmaTuttecompo}\label{lem:Tutte-copositional}
	Let $k$ be any positive integer. There exists a mapping $\varphi$ from $k$-sourced graphs to $\mathbb{Z}[x,y]$ such that
	\begin{enumerate}[(i)]
		\item $\varphi(G)$ is equal to Tutte polynomial of $G$ when $G$ has no sources,
		\item $\varphi(\forget_i(G))$ is a polynomial function of $\varphi(G)$, for all sourced graphs $G$,
		\item $\varphi(\join(G,H))$ is a polynomial function of $\varphi(G), \varphi(H)$, for sourced graphs $G,H$ that have at most one common source.
	\end{enumerate}
\end{restatable}
%Remark: W pracy Andrzejaka ''An algorithm for the Tutte polynomials of graphs of bounded treewidth '' bardzo możliwe, że jest dowód kompozycjonalności w ogólnym przypadku.
\begin{restatable}{fact}{factTuttedistinguishing}\label{fact:Tutte-distinguishing-power}
	$T(G) = x^n$ for all graphs $G$ of treewidth 1 with $n$ edges.
\end{restatable}
Let $\charact(G)$ denote characteristic polynomial of a graph $G$.
\begin{restatable}{lemmarestat}{lemmacharcompo}\label{lem:characteristic-compositional}
	Let $k$ be any positive integer. There exists a mapping $\varphi$ from $k$-sourced graphs to some power of $\mathbb{Z}[t]$ such that
	\begin{enumerate}[(i)]
		\item first coordinate of $\varphi(G)$ is equal to characteristic polynomial of $G$ when $G$ has no sources,
		\item $\varphi(\forget_i(G))$ is a polynomial function of $\varphi(G)$, for all sourced graphs $G$,
		\item $\varphi(\join(G,H))$ is a polynomial function of $\varphi(G), \varphi(H)$, for sourced graphs $G,H$ that have at most one common source.
	\end{enumerate}
\end{restatable}
\begin{fact}\label{fact:characteristic-distinguishing-power}\cite{schwenkCospectral73}
	For almost every graph $G$ of treewidth 1, in the sense of probability, there exists a graph $G'$ of treewidth 1 such that
	$
		\charact(G) = \charact(G').
	$
\end{fact}

Let us mention a related paper \cite{contrerasGluingLaplacians20}, which studies characteristic polynomial and graph Laplacians (which are determinants of other matrices associated to a graph) of fusion of graphs, which is denoted \emph{interface gluing} in the paper.
\subsection{Restriction to subclassess}\label{subsec:subclassess}
Another direction can be to try to find a class of graphs \Cc and a graph polynomial which is a compositional complete isomorphism invariant on \Cc. In this subsection we make such attempt; we test an example class and (an infinite) polynomial, with a negative result however.

In order to begin our considerations, we note that notions of: \mso-transduction, sourced graph algebra and walk series can be generalised to edge-labelled graphs. Many useful facts that hold in unlabelled case remain true, in particular, compositionality of walk series.

The example class we consider is one of edge-labelled cycles. The choice of class is motivated by the following problem: 
	\decisionproblem{cyclic shift-equivalence for functional string-to-string \mso transductions.}
	{functional string-to-string \mso transductions $\Tt_1,\Tt_2$.}
	{is it the case that for every input $w$, the two outputs  $\Tt_1(w),\Tt_2(w)$ are equivalent up to cyclic shift?}
\iffalse
\mso string-to-string transductions can be defined by string-to-($\Sigma^*, \cdot$) register transducers (fill-in reference), and in consequence, when considered modulo cyclic shift-equivalence, they can be seen as string-to-\algcyclic register transducers, where \algcyclic is the following (sorted) algebra: (fill-in czy używać sortów. Są one przydatne też przy dowodzie kompozycjonalności szeregu ścieżkowego, ale sam nie jestem przekonany)
\algebradefinition{
	$\algcyclic_1$: (sort 1 ) words of positive length over alphabet $\Sigma$,\\
	$\algcyclic_0$: (sort 0 ) cyclic words of positive length, labelled by alphabet $\Sigma$.
}
{
	$\cdot : \algcyclic_1 \times \algcyclic_1 \to \algcyclic_1$ - concatenation of words.\\
	$\cyclic:\algcyclic_1 \to \algcyclic_0$ - maps a word to cyclic word it represents.\\
}
\fi
Let us remark that above transductions can be seen as \mso string-to-graph transductions, with output of treewidth 2 (and even pathwidth 2). This is done by applying to outputted labelled cycles a certain ''delabelling'' transduction, i.e. an injective \mso transduction of edge-labelled cycles into unlabelled graphs. % (bo ta algebra jest symulowalna pathwidth 2, w następujący sposób:) %słowo reprezentujemy jako dwie poziome kreski z wyznaczoną ilością widełek. Grafy są nieskierowane, więc słowo zapisujemy ze znacznikami końców. Użycie znaczników z kolei powoduje użycie separatorów, bo przez znaczniki niemożliwa jest konkatencja ''zwyczajnie'' przez sklejanie, tylko tutaj znacznik końca zostanie zastąpiony separatorem - dajmy, że znacznik końca ma 2 widełki, a separator 3 widełki, to taka zamiana będzie wymagała po prostu dodania widełki.
%Opis symulacji
%Stałe:
%	wprowadzimy teraz pojęcie litery, ale uwaga: litery z algebry \algcyclic nie będą mapowane na litery, tylko na ich konkatenacje.
	%litera to jest graf \bullet - \bullet - \bullet, gdzie ze środkowego wierzchołka wychodzi $k$ widełek, $k >= 1 $.
%	znacznik poczatku: 1 widełka
%	znacznik końca : 2 widełki
%	separator : są to 2 litery, znacznik końca złączony ze znacznikiem początku
%	litera a : 4 widełki
%	litera b : 5 widełek
%	
%	litera z algebry \algcyclic jest mapowana na (znacznik pocz)(litera)(znacznik końc)
%	słowo z algebry \algcyclic jest mapowane na (znacznik pocz)(litera1)(separator)(litera2)...(separator)(litera ostatnia)(znacznik końc)
%Konkatencja - zwyczajne łączenie grafów, końca pierwszego z początkiem drugiego
%Operacja cyclic: sklejenie portu końcowego z początkowym (tego nie ma w algebrze pathwidth/treewidth, ale łatwo to zasymulować, np. przez nigdy nie dopisywanie "ostatniej" krawędzi znacznika końcowego (tego co naprawdę stoi na końcu, te w separatorach zostawiamy normalnie)) i narysowanie jej dopiero w momencie wykonania operacji cyclic.


Throughout this subsection we treat both words and cyclic words as directed edge-labelled paths and cycles; for example, we treat word $aba$ as $\bullet\xra{a}\bullet\xra{b}\bullet\xra{a}\bullet$. This enables us to define walk series for edge-labelled graphs analogously to one defined for unlabelled graphs (see subsection \ref{sec:power-series}).

Let $G$ be a directed edge-labelled graph. Its \emph{walk series} $\eta(G)$ is defined as $$\eta(G) = \sum_{n,m=0}^{\infty}\eta_{n,m}(G)\cdot a^n b^m \in \mathbb{Z}[[a,b]],$$ where $\eta_{n,m}(G)$ is the number of walks in $G$ with $n$ $a$-edges and $m$ $b$-edges. For example, walk series of a cyclic word $ab$ (seen as directed cycle
%zrobienie tego obrazka jest trudne w LaTeX-u
%\adjustbox{scale=0.5,center}{%
%\begin{tikzcd}
%	\bullet\arrow[bend right]{r}[black,swap]{a}  & \bullet \arrow[bend right]{l}[black, swap]{b}
%\end{tikzcd}
%}
) is equal to $a + b + 2ab + a^2b + ab^2+2a^2b^2+\ldots.$
The definition is analogous to unlabelled one in a following sense: if a walk is treated as concatenation of its consecutive edges, walk series is obtained by substituting for each edge in the set of all walks an indeterminate in a ring of polynomials, i.e:
$$
\eta(G) = \sum_{P: P \text{ is path in } G}a^{\#_a(P)}b^{\#_b(P)}.
$$

\iffalse

Throughtout the proof we assume all paths and words to be nonempty.

	We start with compositionality with respect to operation $\cyclic$. We try to define for a word a tuple of series, from which we can reconstruct the walk series of cyclic word it represents with operations $+, \cdot$ and $^+$ -- the Kleene plus (fill-in reference, myślę że potrzebny)) that can be updated with the same operations after concatenating words. Hence we first define the following ''split''/partition (fill-in wybrać) of set of positive-length paths in word $w$ (which are the same as its nonempty subwords (fill-in chyba można usunąć tę uwagę)) 	
 \begin{itemize}
 	\item $w^{()}$ - paths that do not touch any endpoint,
 	\item $w^{(]}$ - paths that touch right endpoint, but do not touch left endpoint,
 	\item $w^{[)}$ - paths that do not touch right endpoint, but touch left endpoint,
 	\item $w^{[]}$ - paths (in fact exactly one), that touch both endpoints.
 \end{itemize}
Now we define $\varphi$ to be a similar ''split'' of $\eta(w)$. Define
$
\varphi(w)$ to be the tuple $(\eta^{lr}(w))_{l,r}
$
where $(l,r)$ are all four possible ''bracketings'' as in definition above and $\eta^{lr}$ is the walk series with summation restricted to paths from $w^{lr}$. 
For example,
$$
\eta^{(]}(w) = \sum_{P: P \text{ is path in } w^{(]}}a^{\#_a(P)}b^{\#_b(P)},
$$
etc. The set of all paths in a cyclic word $\overline{w}$ can be obtained from our partition (fill-in czy może użyć split, dla jedności z oznaczeniem w pozostałych dowodach) by the expression
$
w^{()} + (w^{(]} + 1)(1 + (w^{[]})^+)(1 + w^{[)}) - 1,
$
hence by analogous expression $\eta(w) = \varphi(\cyclic(w))$ can be obtained from $\varphi(w)$. In conseqeunce $\varphi$ is compositional with respect to operation $\cyclic$.\\
Now let us move to concatenation operation.
Recall that we assume words $w,v$ to be of positive length.
We have following equalities
\begin{itemize}
	\item $(wv)^{()} = w^{()} + (1+w^{(]})(1 + v^{[)}) - 1 + v^{()}$ 
	\item $(wv)^{(]} = (w^{(]}+1)v^{[]} + v^{(]}$
	\item $(wv)^{[)} = w^{[)} + w^{[]}(1 + v^{[)})$
	\item $(wv)^{[]} = w^{[]}v^{[]}$.
\end{itemize}
and hence analogous formulas hold for $\eta$. This proves compositionality of $\varphi$ with respect to concatenation, proving that $\varphi$ is compositional.
\fi
Let us discuss $\eta$'s distinguishing power on the class of edge-labelled cycles. First observe, that $\eta$ is the same for any cyclic word and its reverse. But there is even more: there are pairs of cyclic words, one not being reverse of another, for which $\varphi$ is the same, for example:
$
w = bbaabbabaaba,
v = bbaababbabaa.
$
%A proof of $\eta(w)$ being equal to $\eta(v)$ is not that quick by hand and was done by computer. (Recursive formulas (fill-in reference do equations) for concatenation however make the hand calculation quick enough for shorter words.)

%One can think of refinements of walk series. One way is to refine it to ''back-and-forth'' walks, with back-going arrows having a different copy of a label. This is simply a walk series for a graph equipped with additional back-going edges, hence it is compositional too. It has at least the distinguishing power of walk series, because the latter is an evalution of the former with back-going copies of letters put to 0. It is an interesting question if back-and-forth walk series distinguishes the mentioned pair of cyclic words $(w,v)$.

%\section{Poprawki}
%Sorted algebra przydałaby się przy dowodzie kompozycjonalności wielomianu charaterystycznego.% Przy $b$-entries: funkcja użyta do update'u jest inna w zależności od tego, którym sourcem sklejamy, jej inność polega na braniu innych elementów krotki. Jest tak też w innych miejscach w characteristic polynomial. Do tego przydałaby się sorted algebra i automaty nad sorted algebrami.

%Z ciekawości: jak własność ''ma treewidth <=k'' zdefiniować w mso? (da się, bo to klasa minor-closed)

%Wyjaśnić bardziej kwestię joinowanie wzdłuż co najwyżej jednego wierzchołka, tego, że to ograniczenie da się wprowadzić, bo transducer może pamiętać w stanach, dla każdego rejestru jakie sources są zdefiniowane w grafie w nim przechowywanym. Może dopisać, że to czy transducer robi join względem co najwyżej jednego źródła można decydować (i to w łatwy sposób, bo to jest sprawdzanie register transducerów o wartościach w algebrze skończonej, czyli tak naprawdę, jak to zostało zauważone w przykładzie z algebrą boolowską, języków regularnych)?

%Można też by sprawdzić, czy faktycznie jest tak, że nie ma wielomianowej funkcji na characteristic polynomial od join wzdłuż dwóch wspólnych wierzchołków. Tam w pracy jest napisane, że jest taka formuła, ale mi się wydaje, że nie ma.



%A third direction of future research might be to consider matrix version of this polynomial.

%%formalny zapis
%\begin{itemize}
%	\item $\eta^{()}(wv) = \eta^{()}(w) + (1+\eta^{(]}(w))(1 + \eta^{[)}(v)) - 1 + \eta^{()}(v)$ ,
%	\item $\eta^{(]}(wv) = (\eta^{(]}(w)+1)\eta^{[]}(v) + \eta^{(]}(v)$,
%	\item $\eta^{[)}(wv) = \eta^{[)}(w) + \eta^{[]}(w)(1 + \eta^{[)}(v))$,
%	\item $\eta^{[]}(wv) = \eta^{[]}(w)\eta^{[]}(v)$.
%\end{itemize}
%Jedyna literatura warta uwagi, którą udało mi się znaleźć to ''Negami, Polynomial Invariants of Graphs'' oraz ''Polynomial invariants of graphs II Seiya Negami & Katsuhiro Ota '' przeczytam je i zobaczę co da się z nich napisać.