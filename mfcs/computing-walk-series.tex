\subsection{Computing walk series}
In this section, we give a proof of Lemma~\ref{lem:compute-power-series}.
\lemcomputePowerSeries*
\begin{proof}[Proof of Lemma~\ref{lem:compute-power-series}]
	
	% The proof is an adaptation of a weighted version of the usual ''DFA to regular expression conversion'' (i.e. weighted automaton to rational power series), with state elimination corresponding to source forgetting.
	
	% Define an \emph{internal walk} to be a walk that does not visit any sources,  with the possible exception of the first and last vertices in the walk, which are allowed to be sources. 
	Define the \emph{source number} of a vertex in a $k$-sourced graph to be  $i \in \set{1,\ldots,k}$ if the vertex is the $i$-th source, and $0$ otherwise. For
	$i,j \in \set{0,\ldots,k}$,
	define an $(i,j)$-walk to be a nonempty walk where the first vertex has source number $i$, the last vertex has source number $j$, and the remaining vertices in the walk are not sources. For a $k$-sourced graph $G$, define 
	\begin{align*}
	\varphi_{ij}(G) \in \aalg
	\end{align*}
	to be the power series where the $n$-th term is the number of nonempty $(i,j)$-walks of length $n$. The constant term  of this power series is zero, since we only consider nonempty walks. Define 
	\begin{align*}
	\varphi(G) \in \aalg^{\set{0,\ldots,k}^2}
	\end{align*}
	to be the tuple of all of the values returned by the functions $\varphi_{ij}$. 
	The main observation is that $\varphi$ is an algebra homomorphism, in the following sense.
	\begin{claim}
		For every term operation $f$ 
		in the algebra of $k$-sourced graphs (see the diagram below for its type), there is a term operation $f^\varphi$ in the algebra $\aalg$ (see the diagram for its type) which makes the following diagram commute
		\begin{align*}
		\xymatrix{
			(\text{$k$-sourced graphs})^X 
			\ar[r]^f
			\ar[d]_{\varphi^X}
			&
			(\text{$k$-sourced graphs})^Y
			\ar[d]^\varphi
			\\
			\aalg^{\set{0,\ldots,k}^2 \times X}
			\ar[r]_{f^\varphi}
			&
			\aalg^{\set{0,\ldots,k}^2 \times Y}
		}
		\end{align*}
	\end{claim}
	Before proving the claim, we use it to finish the proof of the lemma. Suppose that $\Aa$ is a register transducer over the algebra of $k$-sourced graphs. The register transducer  $\Bb$, which is over the algebra $\aalg$, has the same input alphabet, and its registers are triples 
	\begin{align*}
	(i,j,r) \qquad \text{where $i,j \in \set{0,\ldots,k}$ and $r$ is a register of $\Aa$.}
	\end{align*}
	The register updates of $\Bb$ are defined by applying the homomorphism from the claim  to the register updates of $\Aa$. 
	The output register of $\Bb$ is $(0,0,r)$, where  $r$ is the output register of $\Aa$.  From the claim it follows 
	\begin{align*}
	\text{$\Aa$ can output $G$ on input $t$} \qquad \iff \qquad 
	\text{$\Bb$ can output $\varphi_{00}(G)$ on input $t$}.
	\end{align*}
	Since $\varphi_{00}(G)$ is the walk series of $G$ when $G$ has no sources, the lemma follows. It remains to prove the claim.
	
	\begin{proof} 
		It is enough to prove the claim for the case when $f$ is one of the basic operations in the algebra of $k$-sourced graphs. In particular, $Y$ has one element. Here the proof is similar to conversion of a nondeterministic automaton into a regular expression, as in the Kleene Theorem.
		
		\begin{itemize}
			\item Suppose that $f$ is a constant, which represents a  $k$-sourced graph where all vertices are sources. Then we have 
			\begin{align*}
			\varphi_{ij}(f) = \begin{cases}
			x & \text{if there is an edge from $i$ to $j$ in $f$}\\
			0 & \text{otherwise.}
			\end{cases}
			\end{align*}
			\item Suppose that $f$ is the  basic operation which inputs two $k$-sourced graphs and outputs their fusion.  
			An $(i,j)$-walk in the fusion $f(G_1,G_2)$ is either an $(i,j)$-walk in $G_1$ or an $(i,j)$-walk in $G_2$, this is because vertices between endpoints cannot be sources. Furthermore, these cases are disjoint, since we are working with graphs that have parallel edges.
			This leads to the following equation for fusions:
			\begin{align*}
			\varphi_{ij}(f(G_1, G_2)) = \varphi_{ij}(G_1) + \varphi_{ij}(G_2).
			\end{align*}
			The right hand side uses component-wise addition, which is a basic operation in $\aalg$.
			\item The last case is when $f$ is a unary operation which forgets  source $c \in \set{1,\ldots,k}$.  Let then $G$ be a $k$-sourced graph. A nonempty $(i,j)$-walk in $f(G)$ is either a nonempty $(i,j)$-walk in $G$, or it has the form
			\begin{enumerate}
				\item a nonempty $(i,c)$-walk in $G$; followed by 
				\item finitely many (possibly zero) nonempty  $(c,c)$-walks in $G$; followed by
				\item a nonempty $(c,j)$-walk in $G$.
			\end{enumerate}
			This leads to the following equation:
			\begin{align*}
			\varphi_{ij}(f(G)) = \varphi_{ij}(G)) +  \varphi_{ic}(G) \times (1+ \varphi_{cc}(G)^+)  \times \varphi_{cj}(G).
			\end{align*}                
%			\qedhere
		\end{itemize}
	\end{proof}
\end{proof}

