
\section{Equivalence  modulo a certain equivalence relation}\label{sec:equivalence-modulo}
In this section, we present our main result, which says  that equivalence is decidable for graph-to-graph transducers of bounded treewidth, modulo a certain equivalence on the output graphs. This equivalence is a relaxation of isomorphism.  

\smallparagraph{Counting homomorphisms} The equivalence on output graphs is based on a paper by Grohe, Dell and Rattan~\cite{groheDellRattan2018}, which gives examples of how various equivalence relations on graphs, notably Weisfeiler-Leman equivalence, can be characterised in terms of counting homomorphisms. 

Define a \emph{homomorphism} from a graph $G$ to a graph $H$ to be any function $h$ from vertices of $G$ to vertices of $H$ which preserves edges in one direction: if there is an edge from $v$ to $w$ in $G$, then there is an edge from $h(v)$ to $h(w)$ in $H$. We write $\homset(G,H)$ for the set of homomorphism from $G$ to $H$. A seminal result of  Lov\'asz~\cite[p.~326]{lovasz1967operations} says that two graphs are isomorphic if and only if they admit the same number of homomorphisms from every graph:
\begin{align*}
\underbrace{G \simeq H}_{\text{isomorphism}} \qquad \text{iff} \qquad  |\homset(F,G)|=|\homset(F,H)| \ \text{for every graph $F$.}
\end{align*}
In~\cite{groheDellRattan2018},
Grohe, Dell and Rattan show that if one restricts the  class of graphs from which $F$ is taken, then one gets various well-studied relaxations of isomorphism. For example, if we only count homomorphisms from graphs $F$ of treewidth at most $k$, then the arising equivalence relation on graphs is Weisfeiler-Leman equivalence of dimension $k$.  The result of~\cite{groheDellRattan2018},  that we use in this paper counts homomorphisms from paths.

\begin{theorem}\label{thm:grohe}(\cite[Theorem 2]{groheDellRattan2018})
    For every  graphs $G_1,G_2$, the following  are equivalent:
    \begin{enumerate}
        \item for every $i \in \set{0,1,\ldots}$, if $P_i$ is the path of length $n$ then
        \begin{align*}
        |\homset(P_i,G_1) | = |\homset(P_i,G_2)|.
        \end{align*}
        \item $G_1$ and $G_2$  have the same number of vertices, say $n$, and there is an $n \times n$ square matrix $X$ with real coefficients such that:
        \begin{enumerate}
            \item each row in $X$ sums up to 1
            \item each column in $X$ sums up to 1
            \item $A_1 X = X A_2$, where $A_i$ is the incidence matrix of graph $G_i$.
        \end{enumerate}
    \end{enumerate}
\end{theorem}
Note that a homomorphism from a path of length $n$ is the same thing is a walk of length $n$, i.e.~a sequence of $n+1$ vertices connected by edges, possibly with repetitions.  In other words, condition 1 above says that for every $n$, the two graphs have the same number of walks of length $n$. The number of walks of length 0 is the number of vertices, and hence condition 1 implies that the graphs have the same number of vertices.     

Condition 2 in the above theorem can be seen as a relaxation of isomorphism in the following sense. 
If in condition 2 we add the requirement that all coefficients in $X$ are non-negative (real or rational, does not make a difference), then we get  \emph{fractional isomorphism}, which is characterised by homomorphisms from trees~\cite[Theorem 1]{groheDellRattan2018}. If we add the requirement that all coefficients in $X$ are non-negative integers, then we get isomorphism.  

\smallparagraph{Decidable equivalence, modulo counting homomorphisms from paths} 
In the following theorem, we show that equivalence for graph-to-graph \mso transductions of bounded treewidth is decidable, when output graphs are identified modulo the equivalence relation from Theorem~\ref{thm:grohe}. 
% We would have preferred to prove decidability of equivalence modulo isomorphism, but so far we have only managed to apply our technique to the equivalence relation from Theorem~\ref{thm:grohe}.
\begin{theorem}\label{thm:path-equivalence}
    Let $\sim$ be the equivalence relation on graphs described in Theorem~\ref{thm:grohe}. 
    The following problem is decidable:
    \decisionproblem{$\sim$-functionality for graph-to-graph \mso transductions of bounded treewidth.}{ $\ell, k \in \set{1,2,\ldots}$ and a graph-to-graph \mso transductions $\Tt$.}{is there some input of treewidth at most $\ell$, which can produce two outputs of treewidth at most $k$ that are not equivalent under $\sim$?}
\end{theorem}
% By Lemma~\ref{lem:transduction-to-registers}, the above problem reduces to deciding $\sim$-functionality for nondeterministic tree-to-$\aalg$ register transducers, where $\aalg$ is the algebra of $k$-sourced graphs. Here, $\sim$-functionality means that output graphs are identified modulo $\sim$. 
 
For the same reasons as in Fact~\ref{fact:equi-decidable}, the above problem is equi-decidable with the $\sim$-equivalence problem, where one asks if two functional transductions produce $\sim$-equivalent outputs, under assumption of bounded treewidth. 

The rest of this section is devoted to showing decidability for the  $\sim$-functionality problem. We first show how a register transducer can compute a representation of the output graph modulo $\sim$ in terms of a power series (Section~\ref{sec:power-series}), and then we show that equivalence is decidable for register transducers which manipulate such power series (Section ~\ref{sec:pol-transducers-any-ring} and Section~\ref{sec:decide-power-series}). Combining this with the reduction to register transducers from Lemma~\ref{lem:transduction-to-registers}, we get the theorem.

\subsection{Computing the power series for walks in a graph}
\label{sec:power-series}
Define a \emph{univariate formal power series over a semiring $\Ring$} to be  sequence of semiring elements $a_0,a_1,\ldots$ which is represented as an infinite univariate polynomial
\begin{align*}
 A = a_0 + a_1x^1 + a_2x^2 + \cdots.
\end{align*}
We simply say power series, instead of univariate formal power series.
We use the name $i$-th term for $a_i$; and we call $a_0$ the constant term.
We write $\powerseries{\Ring}$ for the  power series over a semiring $\Ring$; this is a semiring. Among all power series, we are particularly interested in the rational ones: a power series is called \emph{rational} if it can be generated from finite power series (a finite power series is one where all but finitely many terms are zero) using the semiring operations $+$ and $\times$ of power series, as well as the following unary  operation called \emph{Kleene plus}:
\begin{align*}
 A^+ \eqdef A + A^2 + A^3 + \cdots,
\end{align*}   
which  is defined only when the constant term   in $A$ is zero. The assumption on a zero constant term ensures that the infinite sum in $A^+$ is well defined: under this assumption, the $i$-th term in $A^j$ with $j>i$ is zero. 


For a graph $G$, define its \emph{walk series} to be the power series 
\begin{align*}
  a_1 x^1 + a_2x^2 + \cdots \qquad \text{where $a_i$ is the number of walks in $G$ of length $i$.}
\end{align*}
By definition, two graphs with the same number of vertices are $\sim$-equivalent if and only if they have the same walk series. 
The first step in the proof of Theorem~\ref{thm:path-equivalence} is the following lemma, which says that the walk series is always rational, and can be computed by a register transducer.
\begin{lemma}\label{lem:compute-power-series}    
    Define  $\aalg$ to be the  algebra where
    \begin{description}
        \item[Domain:]power series in $\powerseries \Nat$ which are rational and have a  zero constant term;
        \item[Operations:] $+$, $\times$, Kleene plus and constants $1$ and $x$.
    \end{description}
    For every $k \in \set{1,2,\ldots}$ and every nondeterministic  \treetotreewidth{ k} register transducer $\Aa$, there is a nondeterministic tree-to-$\aalg$ register transducer $\Bb$ which makes the following diagram commute (arrows are binary relations):
    \begin{align*}
    \xymatrix@C=1cm{
       &  \trees \Sigma   
        \ar[dr]^{\Bb}
        \ar[dl]_{\Aa} \\
        \txt{graphs of
        treewidth $< k$} \ar[rr]_-{G \mapsto \text{walk series of $G$}}& & \natx
    }
    \end{align*}
\end{lemma}
\begin{proof}
    
    %J
    
        % The proof is an adaptation of a weighted version of the usual ''DFA to regular expression conversion'' (i.e. weighted automaton to rational power series), with state elimination corresponding to source forgetting.
    
        
        % Define an \emph{internal walk} to be a walk that does not visit any sources,  with the possible exception of the first and last vertices in the walk, which are allowed to be sources. 
        Define the \emph{type} of a vertex in a $k$-sourced graph to be  $i$ if the vertex is the $i$-th source for $i \in \set{1,\ldots,k}$, and $0$ otherwise. For types
        $i,j \in \set{0,\ldots,k}$,
        define an  $(i,j)$-walk to be a walk where the first vertex has type $i$, the last vertex has type $j$, and the remaining vertices in the walk are not sources. Define the $(i,j)$-walk series of a $k$-sourced graph, denoted by $\varphi_{ij}(G)$, to be the power series where the $n$-th term is the number of nonempty $(i,j)$-walks of length $n$. (In particular, the constant term  of this power series is zero, since we only consider nonempty walks.) Finally, define 
        \begin{align*}
        \varphi_{ij} : \text{$k$-sourced graphs} \to \aalg^{\set{0,\ldots,k}^2} \qquad G \mapsto (\varphi_{ij}(G))_{i,j \in \set{0,\ldots,k}}.
        \end{align*}
        The main observation is that $\varphi$ is an algebra homomorphism, in the following sense.
        \begin{claim}
            For every term operation $f$ 
            in the algebra of $k$-sourced graphs, there is a term operation $f^\varphi$ in the algebra $\aalg$ which makes the following diagram commute
            \begin{align*}
            \xymatrix{
                (\text{$k$-sourced graphs})^X 
                \ar[r]^f
                \ar[d]_{\varphi^X}
                &
                 (\text{$k$-sourced graphs})^Y
                \ar[d]^\varphi
                \\
                \aalg^{\set{0,\ldots,k}^2 \times X}
                \ar[r]_{f^\varphi}
                &
                \aalg^{\set{0,\ldots,k}^2 \times Y}
            }
            \end{align*}
        \end{claim}
        Before proving the claim, we use it to finish the proof of the lemma. Suppose that $\Aa$ has registers $\Ring$. The automaton $\Bb$ has the same input alphabet, and registers $\set{0,\ldots,k}^2  \times R$.  The register updates of $\Bb$ are defined by applying the operation $f \mapsto f^\varphi$ to the register updates of $\Aa$. 
        If $\Ring$ is the output register of $\Aa$, then the output register of $\Bb$ is $(0,0,r)$.  From the claim it follows 
        \begin{align*}
        \text{$\Aa$ can output $G$ on input $t$} \qquad \iff \qquad 
        \text{$\Bb$ can output $\varphi_{00}(G)$ on input $t$}.
        \end{align*}
        Since $\varphi_{00}(G)$ is the walk series of $G$ when $G$ has no sources, the lemma follows. It remains to prove the claim.

        \begin{proof} 
            It is enough to prove the claim for the case when $f$ is one of the basic operations in the algebra of $k$-sourced graphs. In particular, $Y$ has one element. Here the proof is similar to conversion of a nondeterministic automaton into a regular expression, as in the Kleene Theorem.
            
            \begin{itemize}
                \item Suppose that $f$ is a constant, which represents a  $k$-sourced graph where all vertices are sources. Then we have 
                \begin{align*}
                \varphi_{ij}(f) = \begin{cases}
                    x & \text{if there is an edge from $i$ to $j$ in $f$}\\
                    0 & \text{otherwise.}
                \end{cases}
                \end{align*}
                \item Suppose that $f$ is the  basic operation which inputs two $k$-sourced graphs and outputs their fusion.  
                An $(i,j)$-walk in the fusion $f(G_1,G_2)$ is either an $(i,j)$-walk in $G_1$ or an $(i,j)$-walk in $G_2$ -- this is because vertices between endpoints cannot be sources. These cases are disjoint, except the hypothetical walk of length 1 from $i$ to $j$. %Furthermore, these cases are disjoint, since we are working with graphs that have parallel edges.
  		This leads to the following equation for fusions:
                \begin{align*}
                \varphi_{ij}(f(G_1,G_2)) = \begin{cases}
                	\varphi_{ij}(G_1) + \varphi_{ij}(G_2) - x, \text{ if both $G_1$ and $G_2$ have an edge $(i, j)$}\\
                	\varphi_{ij}(G_1) + \varphi_{ij}(G_2), \text{ otherwise}.
                \end{cases}
                \end{align*}
                The right hand side uses component-wise addition, which is a basic operation in $\aalg$.
                \item The last case is when $f$ is a unary operation which forgets  source $c \in \set{1,\ldots,k}$.  Let then $G$ be a $k$-sourced graph. A nonempty $(i,j)$-walk in $f(G)$ is either a nonempty $(i,j)$-walk in $G$, or it has the form
                \begin{enumerate}
                    \item a nonempty $(i,c)$-walk in $G$; followed by 
                    \item finitely many (possibly zero) nonempty  $(c,c)$-walks in $G$; followed by
                    \item a nonempty $(c,i)$-walk in $G$.
                \end{enumerate}
                This leads to the following equation:
                \begin{align*}
                \varphi_{ij}(f(G)) = \varphi_{ij}(G)) +  \varphi_{ic}(G) \times (1+ \varphi_{cc}(G)^+)  \times \varphi_{cj}(G).
                \end{align*}                
            \end{itemize}
        \end{proof}
        
\end{proof}
\subsection{Equivalence for register transducers over a computable field}\label{sec:pol-transducers-any-ring}

%(szkic wstępu) Equivalence of tree-to-($\Sigma^*, \cdot$) register transducers has been proved in \cite{seidlManethKemper2018}. The proof goes by simulating them with tree-to-\Z \emph{polynomial transducers}, which are a kind of register transducers that we define in next paragraph, and showing decidability of equivalence for them. To prove the main theorem we will use decidability of equivalence of tree-to-\Ring polynomial transducers, where \Ring is a ring of rational power series.

In this section, we prove Theorem \ref{thm:equivalence-for-computable-ring}.%, followed by a proof of Theorem .
\thmequivalenceComputableRing*
First we give definitions concerning computable rings. Later we sketch a proof from \cite[Theorem 6.6]{seidlManethKemper2018} of decidability of equivalence for register transducers over field \Q, from which follows the same results for \Z. Later we show that the same proof works for any computable field (Lemma \ref{lem:equivalence-for-computable-field}), from which Theorem \ref{thm:equivalence-for-computable-ring} follows.

\smallparagraph{Computable rings and fields}
By a ring we always mean a \emph{commutative} ring.
We say that a ring $\Ring$ is \emph{computable} if its elements can be enumerated in a way that addition, multiplication (and equality test, if the enumeration is not unique) are computable functions. %A \emph{nonzero} element $r$ of $\Ring$ is called a \emph{zero divisor} if $r \cdot s = 0$ for some $s \neq 0$.
We view field as an algebra, with addition, multiplication, and constants for every element; a field is computable if it is computable as a ring.

%Following \cite{seidlManethKemper2018}, by a \emph{polynomial transducer} we mean any register transducer over an algebra which is a ring.

\smallparagraph{The proof for \Z}
We sketch the proof of decidability of equivalence of \polTsover{\Q}, following \cite{seidlManethKemper2018}.

%Ring \Z can be embedded into field \Q and the proof is done for \polTsover{\Q}.
The proof goes by overapproximation, for each control state $q \in Q$, the set of configurations reachable in $q$ by a set of solutions of a system of polynomial equations (i.e. Zariski closed set).
$Q$-tuple of (polynomial) ideals that represent those systems forms an \emph{inductive invariant}.
The main algorithm runs two semiprocedures in parallel -- one searching for a counterexample to equivalence, one searching for an inductive invariant that certifies equivalence. The algorithm can be formulated for any ring, %\footnote{And for any algebra in fact. However, we do not discuss this further due to lack of place.}
but it works for field \Q for the following reasons.

The certificate of equivalence (inductive invariant) is a tuple of ideals that satisfy certain inclusions. Testing if those inclusions hold is done by Buchberger's algorithm. Every ideal can be finitely represented due to Hilbert's Basis Theorem and in consequence, tuples of them can be enumerated.
The proof of existence of inductive invariants (for a pair of equivalent polynomial transducers) uses \cite[Lemma 6.3]{seidlManethKemper2018}, let us denote it \emph{Product Lemma}, about ideals of polynomials that zero on cartesian products of sets.

\smallparagraph{Generalisation to other rings} Hilbert's Basis Theorem and Product Lemma hold for any field, with the same proof as for \Q (in fact, this is stated explicitly in \cite{seidlManethKemper2018} before Product Lemma). Buchberger's algorithm works for any computable field as well. We therefore obtain the following Lemma.

\begin{lemma}\label{lem:equivalence-for-computable-field}
Let $\Field$ be a computable field, considered as an algebra with addition and multiplication operations, with constants being the whole universe. Then equivalence is decidable for register transducers over $\Field$.
\end{lemma}

In consequence equivalence is decidable for any computable ring that can be embedded into a field. %; rational power series can be embedded into field of rational functions the following way.
%
%Let $A$ be a proper rational power series. Then $A^+ \cdot (1-A) = A$, hence $A^+$ can be seen as a fraction of polynomials of $A$; by induction, all proper rational power series are fractions of polynomials, which yields the embedding. Formally, a finite power series, i.e.~a polynomial in $\Int[x]$ can be seen as a rational function with denominator 1 and this mapping can be extended to a function $\alpha : \aalg \to \Int(x)$ on all proper rational series 
%which is defined by:
%\begin{align*}
%\alpha(A \times B) = \alpha(A) \times \alpha(B) \quad \alpha(A + B) = \alpha(A) + \alpha(B) \quad \alpha(A^+) = \frac{\alpha(A)}{1-\alpha(A)}.
%\end{align*}
%One can show that the above definition gives indeed a function, which is furthermore injective. By definition $\alpha$ is a semiring homomorphism, which finishes the construction of embedding. 
% Since all operations in the algebra $\aalg$ correspond to term operations in $\Int(x)$ augmented with division operation, it follows that equivalencse of register transducers over the algebra $\aalg$ reduces to equivalence for register transducers over the field $\Int(x)$, whose register updates may use division.
%Therefore we obtain the following result.
%%\begin{proof}
%%	Like every commutative ring without zero divisors, $\Ring$ embeds to a field (the embedding is an injective ring homomorphism). Since $\Ring$ is computable, then the field is also computable. 
%%\end{proof}
%From this the Corollary \ref{cor:equivalence-pol-transducers-Zrat} follows.
%\corEquivPolTransducersZrat*
%\begin{proof}%[Proof of Corollary \ref{cor:equivalence-pol-transducers-Zrat}]
%	This is an application of Therem \ref{thm:equivalence-polynomial-automata-over-a-ring} -- 
%It suffices to observe that both ring of polynomials and ring \Zrat are commutative and have no zero divisors.
%\end{proof}
%
%In fact, \Zrat can be embedded to \emph{some} field for a more abstract reason (the above construction is a proof of its computability though).
Such rings can be characterised the following way.
\begin{lemma}
\label{lem:ring-with-no-zero-divisors-subring-of-a-field}
		Let $\Ring$ be a ring. Then
		\begin{enumerate}[(i)]
			\item $\Ring$ can be embedded into a field if and only if it has no zero divisors,
			\item if $\Ring$ is computable then the embedding can be done effectively.
		\end{enumerate}
\end{lemma}
\begin{proof}
	(Computable) ring $\Ring$ with no zero divisors can be (effectively) embedded into its field of fractions.
\end{proof}
We conclude with the proof of Theorem \ref{thm:equivalence-for-computable-ring}.
\begin{proof}[Proof of Theorem \ref{thm:equivalence-for-computable-ring}]
	Combine Lemma \ref{lem:ring-with-no-zero-divisors-subring-of-a-field} with Lemma \ref{lem:equivalence-for-computable-field}.
\end{proof}
%We finish this section with proof of Theorem \ref{thm:equivalence-with-division}.
%\thmequivalenceWithDivision*
%\begin{proof}
%(fill-in proof)
%\end{proof}
%\begin{remark}
%Product Lemma is used in the proof only if transducers' input alphabet contains a symbol of rank larger than 1. Therefore for \emph{string}-to-$\Ring$ transducers, Theorem generalises to rings that have finite representations of ideals and have decidable ideal inclusion -- there are examples of such rings, which are not a subring of some field.
%\end{remark}

\subsection{Equivalence for register transducers that compute walk series}\label{sec:decide-power-series}
By Lemmas~\ref{lem:transduction-to-registers} and~\ref{lem:compute-power-series}, the  equivalence problem from Theorem~\ref{thm:path-equivalence} reduces to equivalence for nondeterministic tree-to-$\aalg$ register transducers, where $\aalg$ is the algebra from Lemma~\ref{lem:compute-power-series}. In this section, we complete the proof the theorem by showing that the latter problem is decidable.
\begin{lemma}\label{lem:functionality-decidable-power-series} If $\aalg$ is the algebra from Lemma~\ref{lem:compute-power-series}, then equivalence is decidable for nondeterministic tree-to-$\aalg$ register transducers. 
\end{lemma}
\begin{proof}
	We will reduce to equivalence of tree-to-\Zrat polynomial transducers, which is shown decidable in Corollary \ref{cor:equivalence-pol-transducers-Zrat}.
	The following equation allows to simulate Kleene plus by division operation; observe that for any $f$ we have $ff^+ + f = f^+$, and in consequence
	$$
	f^+ = \frac{f}{1-f}.
	$$
	This makes tree-to-\aalg register transducer a tree-to-\Zrat polynomial transducer \emph{with division}, i.e. a polynomial transducers with additional feature that register updates can use division operation.
	The rest of the proof is devoted to showing decidability of equivalence of polynomial transducers with division.
	\begin{lemma}
		Equivalence of tree-to-$\Ring$ polynomial transducers with division is decidable, for a ring with no zero divisors $\Ring$.
	\end{lemma}
	\begin{proof}
		We will adapt the construction of projective line.
		Consider the following representation of elements of $\Ring$: a pair $(x,y)\in \Ring^2$ represents $\frac{x}{y} \in \Ring$. This representation is not unique, and is undefined on some pairs. Then addition, multiplication and, importantly, division of elements of $\Ring$ can be realised as polynomial operations on their representations in $\Ring^2$. This is employed in definition of the following algebra \balg:
		%\begin{align*}
		%	&\frac{x}{y} + \frac{z}{t} = \frac{xt+yz}{yt},\\ 
		%	&\frac{x}{y} \cdot \frac{z}{t} = \frac{xz}{yt},\\
		%	&\left(\frac{x}{y}\right)^+ = \frac{\frac{x}{y}}{1-\frac{x}{y}} = \frac{x}{y-x},
		%\end{align*}
		%and define algebra $\balg$ the following way:
		\algebradefinition{$\Ring \times \Ring$}{$+_B, \cdot_B$ of arity 2, $^+$ of arity 1:
			\begin{align*}
			&(x,y) +_B (z,t) = (xt+yz, yt),\\
			&(x,y)\cdot_B(z,t) = (xz, yt),\\
			&(x,y)^{+_B} = (x, y-x).
			\end{align*}}
		
		Also observe that there is a ''zero-testing'' polynomial function $\eqzeroinalgb$ that satisfies
		$$
		\eqzeroinalgb(x,y) = 0 \iff (x,y) \text { represents } 0,
		$$
		that is, $\eqzeroinalgb(x,y) = x$.
		
		Now, for $\Aa$ being tree-to-$\Ring$ polynomial transducer with division, one can make a tree-to-$\balg$ polynomial transducer $\Bb$, using equations above. Then for each input tree $t$, $\Bb(t)$ represent $\Aa(t)$.
		
		Thus the reduction is as follows: for polynomial transducers with division $\Aa_1, \Aa_2$ take polynomial transducer $\Bb_{1,2}$ corresponding to $\Aa_1-\Aa_2$. Then $\Aa_1 \equiv \Aa_2$
		%\Leftrightarrow
		if and only if 
		$\eqzeroinalgb(\Bb_{1,2}) \equiv 0$,
		that is,
		polynomial transducer $\eqzeroinalgb(\Bb_{1,2})$ obtained by composing output function of $\Bb_{1,2}$ with $\eqzeroinalgb$, is equivalent to $0$.
	\end{proof}	
\end{proof}
