\section{Two polynomials: Tutte and characteristic -- revisited}
This section is devoted to proving Lemma \ref{lem:Tutte-copositional}, Fact \ref{fact:Tutte-distinguishing-power}, Lemma \ref{lem:characteristic-compositional} and Fact \ref{fact:characteristic-distinguishing-power}. Prior to that we define Tutte and characteristic polynomial and provide necessary facts about them.
\subsection{Notation}
Let $G,H$ be sourced graphs. If they have no common sources then their $\join$ is their disjoint union, and is denoted by $G\cup H$. If they have exactly one common source, their $\join$ is denoted by $G\sqcup_a H$, where $a$ is the number of the common source. 

An edge is called a \emph{bridge} if removing it results in disconnecting its connected component. We call an edge \emph{ordinary} if it is not a bridge nor a loop (note that graphs in our setting have no loops by assumption). \emph{Contraction} of an edge is defined as deleting it and then identifying its endpoints. For an edge $e$, by $G \setminus e$ we denote a graph resulting from deleting $e$, and by $G/e$ the graph resulting from contracting $e$. For a vertex $v$, by $G-v$ we denote a graph resulting from deleting $v$. By $A_G$ we denote adjacency matrix of $G$, whose rows and columns are indexed by vertices of $G$. By $I$ we denote identity matrix of appriopriate size, which always will be clear from the context. For a matrix $M$, we denote its $(i,j)$-th entry by $M_{i,j}$, its $i$-th row (or simply $i$-row) by $M_{i,\cdot}$ and its $j$-th column (or simply $j$-column) by $M_{\cdot, j}$. By $\widehat{v}$ we represent the \emph{absence} of $v$, for example $M_{i, \widehat{v}}$ represents $i$-th row without its $v$-entry.
We often identify a source number with a vertex it denotes. In particular, we denote graph $G$ with $a$-th source deleted by $G-a$, and if matrix entries are indexed by vertices of a sourced graph, we simply write ''$a$-row'' instead of ''$a$-th source row''.
\subsection{Definitions and proofs}
\emph{Tutte polynomial} is a graph polynomial that encodes numbers of certain patterns in a graph. For example one can read from it number of spanning trees of a given graph and number of $k$-colourings for each $k$. It can be defined by a recursive formula, called deletion-contraction formula (see e.g. \cite{Ellis-Monaghan2011}):
\begin{subequations}
\renewcommand{\theequation}{\theparentequation.\arabic{equation}}
\begin{align}
\indent\indent &T(G) = T(G\setminus e) + T(G/e), \text{ where } e \text{ is an ordinary edge}
\\
\text{otherwise, $G$ consists of } & \text{$i$ bridges and $j$ loops and} \nonumber
\\
&\indent\indent T(G) = x^iy^j. \label{eq:Tutte-definition2}
\end{align}
\end{subequations}
Tutte polynomial admits another characterisation, as a generating function: if $G = (V,E)$, then $$T(G) = \sum_{A\subseteq E} (x-1)^{r(E) - r(A)} (y-1)^{n(A)},$$ where $r(A)$ is the rank function and $n(A)$ is the nullity function -- we omit their definitions, as they will not be used.
Let us now recall and prove Lemma \ref{lem:Tutte-copositional} and Fact \ref{fact:Tutte-distinguishing-power}.
\lemmaTuttecompo*
\begin{proof}
Tutte polynomial satisfies the following equalities (\cite[Proposition 1]{Ellis-Monaghan2011}):
$$
T(G\cup H) = T(G) \cdot T(H), \qquad T(G\sqcup_a H) = T(G) \cdot T(H).
$$
In consequence, one can simply put $\varphi(G) = T(G)$ for any sourced graph $G$. %We don't study the general case where fusions can be performed along arbitrary sets of source-vertices, because Tutte polynomial is not a good candidate anyway -- 
\end{proof}
\factTuttedistinguishing*
\begin{proof}
It follows from deletion-contraction definition of Tutte polynomial -- graph of treewidth 1 with $n$ edges consists of $n$ bridges and 0 loops.
\end{proof}
Let us now study another graph polynomial -- characteristic polynomial of a graph, denoted by $\charact(G)$. \emph{Characteristic polynomial} of a graph $G$ is defined as characteristic polynomial of its adjacency matrix $A_G$, i.e. a polynomial in one variable $t$, equal to $\det(A_G - t \cdot I)$. Interestingly, two graphs have the same characteristic polynomial if and only if they have the same number of length-$n$ \emph{closed walks}, i.e. walks that start and end in the same vertex, for all $n \geq 0$ (cf. ''Counting homomorphisms'' before Theorem \ref{thm:grohe}). 

We will now show that characteristic polynomial is compositional, again with a restriction that fusions are performed along at most one source-vertex. 
\lemmacharcompo*
\begin{proof}
Before defining mapping $\varphi$, let us take a closer look at characteristic polynomial of $G \sqcup_a H$.
Put $G':= G-a, H':=H-a$. By assumption $G'$ and $H'$ are disjoint. 

When picturing matrices, we make certain simplifactions of notation, which technically speaking are a mistake. To remove some symbols, we replace $A_G - t\cdot I$ with $A_G$, and then, to avoid double subscripts, we replace $A_G$ with $G$. For example we write  $G_{a,\cdot}$ instead of $(A_G - t\cdot I)_{a,\cdot}$, $G'$ instead of $A_{G'}$ etc. (This will not cause many confusions --- computing determinant and cofactors of $A_G - t\cdot I$ is very similar to $A_G$, because the former is also an adjacency matrix, but of a graph $G$ with additional self-loops added to each vertex, with a weight $-t$.)

If we order vertices of $G$ and $H$ so that $a$ is the last vertex of $G$ and the first vertex of $H$, adjacency matrix of $G\sqcup_aH$ looks like this: \\
\scalebox{\skalamatadjacency}{
	%\documentclass{article}
%\usepackage{tikz}
%\usepackage{geometry}
%\usepackage{amsmath}
%\usetikzlibrary{matrix,positioning}

%\begin{document}
\begin{tikzpicture}[
%style
stylmacierzy/.style={
	matrix of math nodes,
	every node/.append style={text width=\sizemacierzy, text height=\sizemacierzy,align=center},
	nodes in empty cells,
	left delimiter=[,
	right delimiter=],
}
]
\matrix[stylmacierzy] (matadjacency)
{
	& & & & & & & & \\
	& & & & & & & & \\
	& & & & & & & & \\
	& & & & & & & & \\
	& & & & & & & & \\
	& & & & & & & & \\
	& & & & & & & & \\
	& & & & & & & & \\
	& & & & & & & & \\
};
\draw[]%pionowy prawy po G'
	(matadjacency-1-3.north east) -- (matadjacency-3-3.south east);
\draw[]%poziomy dolny po G'
	(matadjacency-3-1.south west) -- (matadjacency-3-3.south east);
\draw[]%pionowy lewy boczny po H'
	(matadjacency-5-5.north west) -- (matadjacency-5-9.north east);
\draw[]%górny po H'
	(matadjacency-5-5.north west) -- (matadjacency-9-5.south west);	
\draw[loosely dashed]%% poziomy przy Ga
	(matadjacency-3-4.south west) -- (matadjacency-3-9.south east);
\draw[loosely dashed]%poziomy górny przy Ha
	(matadjacency-4-1.south west) -- (matadjacency-4-4.south east);
\draw[loosely dashed]%lewe dolne 0, prawy pionowy
	(matadjacency-4-3.north east) -- (matadjacency-9-3.south east);	
\draw[loosely dashed]%prawe górne 0, lewy pionowy
	(matadjacency-1-5.north west) -- (matadjacency-4-5.south west);	
%
\node[font=\GprimSize] 
	at (matadjacency-2-2) {$G'$};
\node[font=\GaSize] 
	at (matadjacency-2-4) {$G_{\cdot, a}$};
\node[font=\GaSize] 
	at (matadjacency-4-2) {$G_{a, \cdot}$};
\node[font=\large] 
	at (matadjacency-4-4) {$-t$};
\node[font=\GaSize] 
at (matadjacency-4-7) {$H_{a, \cdot}$};
\node[font=\GaSize] 
at (matadjacency-7-4) {$H_{\cdot, a}$};
\node[font=\HprimSize]
	at (matadjacency-7-7) {$H'$};
\node[font=\ZeroSize]%prawe górne 0
	at (matadjacency-2-7) {$0$};
\node[font=\ZeroSize]
	at (matadjacency-7-2) {$0$};
%
%\draw[]
%	(matadjacency-3-1.south) --
%	node[left=41pt] {Laplace}
%	(matadjacency-3-1.south)
%	;
% \draw[]
%	(matadjacency-4-1) --
%	node[left=25pt] {expansion - - -}
%	(matadjacency-4-1)
%	;
\end{tikzpicture}

%\end{document}
}
\\
where distinguished row and column are the ones corresponding to vertex $a$.
Laplace expansion at $a$-row is of the following form (cofactors are replaced by ellipsis for simplicity of notation):
\begin{equation*}
\det(A_{G\sqcup_a H}) = \sum_{v \in (G-a)} G_{a, v} \cdot \ldots + (-t) \cdot \ldots + \sum_{w \in (H-a)} H_{a,w} \cdot \ldots,
\end{equation*}
where $v$-cofactor, $a$-cofactor and $w$-cofactor, for $v\in (G-a), w \in (H-a)$ are determinants of matrices depicted in Figure \ref{fig:v-kofaktor}, Figure \ref{fig:a-kofaktor} and Figure \ref{fig:w-kofaktor} (recall that for any vertex $v$, symbol $\widehat{v}$ represents the \emph{absence} of $v$).

\begin{figure}
	\scalebox{\skalamacierzy}{
		%\documentclass{article}
%\usepackage{tikz}
%\usepackage{geometry}
%\usepackage{amsmath}
%\usetikzlibrary{matrix,positioning}

%\begin{document}
\begin{tikzpicture}[
%style
stylmacierzy/.style={
	matrix of math nodes,
	every node/.append style={text width=\sizemacierzy, text height=\sizemacierzy,align=center},
	nodes in empty cells,
	left delimiter=[,
	right delimiter=],
}
]%to ma juz \begin{tikzpicture}[styl]
	\matrix[stylmacierzy] (matadjacency)
	{
		& & & & & & & \\
		& & & & & & & \\
		& & & & & & & \\
		& & & & & & & \\
		& & & & & & & \\
		& & & & & & & \\
		& & & & & & & \\
		& & & & & & & \\
	};
	\draw[loosely dashed]%pionowy prawy po G'
		(matadjacency-1-2.north east) -- (matadjacency-3-2.south east);
	\draw[]%dolny po G'
		(matadjacency-3-1.south west) -- (matadjacency-3-2.south east);
	\draw[]%poziomy górny po H'
		(matadjacency-3-4.south west) -- (matadjacency-3-8.south east);
	\draw[]%pionowy lewy boczny po H'
		(matadjacency-4-4.north west) -- (matadjacency-8-4.south west);	
	\draw[]% poziomy przy Ga
		(matadjacency-3-3.south west) -- (matadjacency-3-3.south east);
	\draw[loosely dashed]%lewe dolne 0, prawy pionowy
		(matadjacency-4-2.north east) -- (matadjacency-8-2.south east);	
	\draw[]%prawe górne 0, lewy pionowy
		(matadjacency-1-4.north west) -- (matadjacency-3-4.south west);	
	%
	\node[font=\GprimSize] 
		at (matadjacency-2-1.5) {$G'_{\cdot, \widehat{v}}$};
	\node[font=\GaSize] 
		at (matadjacency-2-3) {$G_{\cdot, a}$};
	\node[font=\GaSize] 
	at (matadjacency-6-3) {$H_{\cdot, a}$};
	\node[font=\HprimSize]
		at (matadjacency-6-6) {$H'$};
	\node[font=\ZeroSize]%prawe górne 0
		at (matadjacency-2-6) {$0$};
	\node[font=\ZeroSize]
		at (matadjacency-6-2) {$0$};
	%
	%\draw[]
	%	(matadjacency-3-1.south) --
	%	node[left=41pt] {Laplace}
	%	(matadjacency-3-1.south)
	%	;
	%\draw[]
	%	(matadjacency-4-1) --
	%	node[left=25pt] {v in $G-a$}
	%	(matadjacency-4-1)
	%	;

\end{tikzpicture}

%\end{document}
	}
	\caption{$v$-cofactor}
	\label{fig:v-kofaktor}
	%\end{figure*}
	%\begin{figure*}
	\scalebox{\skalamacierzy}{
		%\documentclass{article}
%\usepackage{tikz}
%\usepackage{geometry}
%\usepackage{amsmath}
%\usetikzlibrary{matrix,positioning}

%\begin{document}
\begin{tikzpicture}[
%style
stylmacierzy/.style={
	matrix of math nodes,
	every node/.append style={text width=\sizemacierzy, text height=\sizemacierzy,align=center},
	nodes in empty cells,
	left delimiter=[,
	right delimiter=],
}
]
	\matrix[stylmacierzy] (matadjacency)
	{
		& & & & & & & \\
		& & & & & & & \\
		& & & & & & & \\
		& & & & & & & \\
		& & & & & & & \\
		& & & & & & & \\
		& & & & & & & \\
		& & & & & & & \\
	};
	\draw[]%pionowy prawy po G'
		(matadjacency-1-3.north east) -- (matadjacency-3-3.south east);
	\draw[]%dolny po G'
		(matadjacency-3-1.south west) -- (matadjacency-3-3.south east);
	\draw[]%poziomy górny po H'
		(matadjacency-3-4.south west) -- (matadjacency-3-8.south east);
	\draw[]%pionowy lewy boczny po H'
		(matadjacency-4-3.north east) -- (matadjacency-8-3.south east);	
	%
	\node[font=\GprimSize] 
		at (matadjacency-2-2) {$G'$};
	\node[font=\HprimSize]
		at (matadjacency-6-6) {$H'$};
	\node[font=\ZeroSize]%prawe górne 0
		at (matadjacency-2-6) {$0$};
	\node[font=\ZeroSize]%lewe dolne 0
		at (matadjacency-6-2) {$0$};
	%
	%\draw[]
	%	(matadjacency-3-1.south) --
	%	node[left=41pt] {Laplace}
	%	(matadjacency-3-1.south)
	%	;
	%\draw[]
	%	(matadjacency-4-1) --
	%	node[left=25pt] {a}
	%	(matadjacency-4-1)
	%	;

\end{tikzpicture}

%\end{document}
	}
	\caption{$a$-cofactor}
	\label{fig:a-kofaktor}
	%\end{figure*}
	%\begin{figure*}
	\scalebox{\skalamacierzy}{
		%\documentclass{article}
%\usepackage{tikz}
%\usepackage{geometry}
%\usepackage{amsmath}
%\usetikzlibrary{matrix,positioning}

%\begin{document}
\begin{tikzpicture}[
%style
stylmacierzy/.style={
	matrix of math nodes,
	every node/.append style={text width=\sizemacierzy, text height=\sizemacierzy,align=center},
	nodes in empty cells,
	left delimiter=[,
	right delimiter=],
}
]%to ma juz \begin{tikzpicture}[styl]
	\matrix[stylmacierzy] (matadjacency)
	{
		& & & & & & & \\
		& & & & & & & \\
		& & & & & & & \\
		& & & & & & & \\
		& & & & & & & \\
		& & & & & & & \\
		& & & & & & & \\
		& & & & & & & \\
	};
	\draw[]%pionowy prawy po G'
		(matadjacency-1-3.north east) -- (matadjacency-3-3.south east);
	\draw[]%poziomy dolny po G'
		(matadjacency-3-1.south west) -- (matadjacency-3-3.south east);
	\draw[]%poziomy górny po H'
		(matadjacency-3-4.south west) -- (matadjacency-3-8.south east);
	\draw[loosely dashed]%pionowy lewy boczny po H'
		(matadjacency-4-5.north west) -- (matadjacency-8-5.south west);	
	%\draw[]% poziomy przy Ga
	%	(matadjacency-3-4.south west) -- (matadjacency-3-4.south east);
	\draw[]%pionowy prawy przy lewe dolne 0, 
		(matadjacency-4-3.north east) -- (matadjacency-8-3.south east);	
	\draw[loosely dashed]%pionowy lewy przy prawe górne 0, 
		(matadjacency-1-5.north west) -- (matadjacency-3-5.south west);	
	%
	\node[font=\GprimSize] 
		at (matadjacency-2-2) {$G'$};
	\node[font=\GaSize] 
		at (matadjacency-2-4) {$G_{\cdot, a}$};
	\node[font=\GaSize] 
	at (matadjacency-6-4) {$H_{\cdot, a}$};
	\node[font=\HprimSize]
		at (matadjacency-6-6) {$H'_{\cdot,\widehat{w}}$};
	\node[font=\ZeroSize, xshift=-4pt]%prawe górne 0
		at (matadjacency-2-7) {$0$};
	\node[font=\ZeroSize]%lewe dolne 0
		at (matadjacency-6-2) {$0$};
	%
	%\draw[]
	%	(matadjacency-3-1.south) --
	%	node[left=41pt] {Laplace}
	%	(matadjacency-3-1.south)
	%	;
	%\draw[]
	%	(matadjacency-4-1) --
	%	node[left=25pt] {w in $H-a$}
	%	(matadjacency-4-1)
	%	;

\end{tikzpicture}

%\end{document}
	}
	\caption{$w$-cofactor}
	\label{fig:w-kofaktor}
\end{figure}


\noindent By the fact that $\det\left(\left|\begin{array}{c|c}
A & \ast\\
\hline
0 & B
\end{array}\right|\right)
=
\det\left(\left|\begin{array}{c|c}
A & 0 \\
\hline
\ast & B
\end{array}\right|\right)
=
\det(A) \cdot \det(B),
$
where $A,B$ are any square matrices, of possibly different sizes,
we get that:
\begin{itemize}
	\item for $v \in G-a$, $v$-cofactor of $G\sqcup_a H$ is equal to $v$-cofactor of $G$ multiplied by $\det(H' - t\cdot I)$, 
	\item $a$-cofactor of $G\sqcup_a H$ is equal to $\det(G' - t\cdot I) \cdot \det(H' - t\cdot I)$,
	\item for $w \in H-a$, $w$-cofactor of $G\sqcup_a H$ is equal to $w$-cofactor of $H$ multiplied by $\det(G' - t\cdot I)$.
\end{itemize}
Now let us define $\varphi(G)$. If $G$ has at least one source-vertex, define $\varphi(G)$ to be a tuple $$\varphi(G):= (\rest{G}{a}, \primecofact{G}{a})_{a \in \sources(G)},$$ where for each source $a$, pair $(\rest{G}{a}, \primecofact{G}{a})$ is a ''split'' of characteristic polynomial of $G$, induced by Laplace expansion of $A_G-t\cdot I$ at $a$-row, defined as follows: (cofactors are replaced again by ellipsis, except of $a$-cofactor, which is denoted by $\primecofact{G}{a}$)
$$
\det(A_G-t\cdot I) = (-t)\cdot \underbrace{\det(A_{G'} - t\cdot I)}_{\primecofact{G}{a}} + \underbrace{\sum_{v \in (G-a)}G_{a,v} \cdot \ldots}_{\rest{G}{a}}.
$$
If $G$ has no sources, define $\varphi(G)$ to be the characteristic polynomial of $G$.

$a$-entries of $\phi(G\sqcup_a H)$ are polynomial functions of $\varphi(G), \varphi(H)$ --
the mentioned ''split'' of characteristic polynomial, now of $G\sqcup_a H$, at $a$-row, is the following:
\begin{align*}
&(-t) \cdot \underbrace{
	\det(A_{G'} - t\cdot I) \cdot \det(A_{H'}-t\cdot I)}_{
	\primecofact{G\sqcup_a H}{a}} +\\
+&\underbrace{\left(\sum_{v\in G-a} G_{a,v} \cdot \ldots \cdot \det(A_{H'} - t\cdot I) \right)
	+
	\left(\sum_{w\in H-a} H_{a,w} \cdot \ldots \cdot \det(A_{H'} - t\cdot I)\right)}_{
	\rest{G\sqcup_a H}{a}},
\end{align*}
in other words
\begin{align*}%nadaję (a właściwie: planuję nadać) numer żeby podkreślić znaczenie tej równości
&\primecofact{G\sqcup_a H}{a} = \primecofact{G}{a} \cdot \primecofact{H}{a},\\
&\rest{G\sqcup_a H}{a} = \rest{G}{a} \cdot \primecofact{H}{a} + \rest{H}{a} \cdot \primecofact{G}{a}.
\end{align*}

Before we continue the proof of compositionality of $\varphi$ w.r.t $\join$ operation, observe that $\varphi$ is immediately compositional with respect to forget operation. If source $i$ is the last remaining source of $G$, $\varphi(\forget_i(G)) = \charact(G) = (-t)\cdot\primecofact{G}{i} + \rest{G}{i}$; if there is more than one source, $\varphi(\forget_i(G))$ is a subtuple of $\varphi(G)$.

$b$-entries of $\varphi(G \sqcup_aH)$, for a source $b$ different than $a$, if there is such, can be computed by similar analysis, this time by multiple-row Laplace expansion taken along a pair of rows: $b$-row and $a$-row and using the fact that $v$-cofactor is the sum of all $(v, v')$-cofactors. %(say, $a$ first, as we have it pictured already)
We get: (first entry is taken from row $b$, second one from row $a$)
\begin{itemize}
	\item for $v \in G' \mathsmaller{\cup} \ \! a, v' \in G' \mathsmaller{\cup} \ \! a, v' \neq v$ each $(v, v')$-cofactor of $G\sqcup_a H$ equals $(v, v')$-cofactor of $G$ multiplied by $\det(H' - t\cdot I)$,
	%\item $a$-cofactor of $A_{G\sqcup_a H}$ equals $\det(G' - t\cdot I) \cdot \det(H' - t\cdot I)$
	\item for $v \in G' \mathsmaller{\cup} \ \! a, w \in H'$ each $(v,w)$-cofactor of $G\sqcup_a H$ equals:
	\begin{itemize}
		\item for $v \in G', w \in H'$, $v$-cofactor of $G'$ multiplied by $w$-cofactor of $H$,
		\item for $v = a, w \in H'$, it equals 0,
	\end{itemize}
	\item for $v \notin (G' \mathsmaller{\cup} \ \! a)$, $v$-entries of $b$-row are equal to 0, hence we do not compute cofactors for them
\end{itemize}
and in consequence,
in row $b$, for each $v\in G-a$, each $v$-cofactor of $G\sqcup_a H$ equals:
\begin{center}
	$v$-cofactor of $G$ multiplied by $\det(H' - t\cdot I)$\\
	+\\
	$v$-cofactor of $G'$ multiplied by sum of (not $a$)-cofactors in $H$, i.e. $\rest{H}{a}$.
\end{center}
From this we get
\begin{align*}
&\primecofact{G\sqcup_a H}{b} = \primecofact{G}{b} \cdot \primecofact{H}{a} + \rest{H}{a} \cdot \primecofact{G'}{b},\\
&\rest{G\sqcup_a H}{b} = \rest{G}{b} \cdot \primecofact{H}{a} + \rest{H}{a} \cdot \rest{G'}{b}.
\end{align*}
Analogous equations hold for sources $c$ of graph $H$. 
This shows that
$
\varphi(G \sqcup_a H)$ can be computed from a tuple: $(\varphi(G), (\varphi(G-b))_{b\in\sources(G)}, \varphi(H), (\varphi(H-c))_{c\in \sources(H)})
$
by a polynomial function.
This does not imply that $\varphi(G)$ is compositional --- however, iterating this observation recursively on sources of $G\sqcup_a H$, implies that $\psi(G)$ defined as
$$\psi(G) := (\varphi(G-b))_{B\subset\sources(G)}$$ is compositional with respect to fusion along one vertex.

Observe that for any sourced graph operation, compositionality of $\varphi$ implies compositionality of $\psi$, hence for the remaining operations it suffices to prove compositionality of $\varphi$.

When graphs being fused have no common sources, that is, when their fuse is disjoint union, in row $b$, $v$-cofactor of $G \cup H$ is equal to $v$-cofactor of $G$ multiplied by characteristic polynomial of $H$, for all $v \in G$, for a $b$-source of graph $G$. In consequence we get
\begin{align*}
\primecofact{G\cup H}{b} = \primecofact{G}{b} \cdot \det(H - t\cdot I), \\
\rest{G\cup H}{b} = \rest{G}{b} \cdot \det(H - t\cdot I).
\end{align*}
In other words, for $b$-source of $G$, $b$-entries of $\varphi(G \sqcup_a H)$ are $b$-entries of $\varphi(G)$ multiplied by characteristic polynomial of $H$, which, as we have seen, is a polynomial function of $\varphi(H)$. The symmetric statement holds for sources of $H$, which finishes the proof of compositionality of $\psi$.
\end{proof}
%\section{Poprawki}
%Dopisać gdzieś, że tu są graph polynomials, więc term operation to jest polynomial function.
%
%Zamiast (v,x)-cofactor w 2-ch miejscach mamy na myśli (v,x)-cofactor przemnozony przez $G_x$