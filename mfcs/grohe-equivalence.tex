
\section{Equivalence  modulo a certain equivalence relation}\label{sec:equivalence-modulo}
In this section, we present our main result, which says  that equivalence is decidable for graph-to-graph transducers of bounded treewidth, modulo a certain equivalence on the output graphs. This equivalence is a relaxation of isomorphism.  

\smallparagraph{Counting homomorphisms} The equivalence on output graphs is based on a paper by Grohe, Dell and Rattan~\cite{groheDellRattan2018}, which gives examples of how various equivalence relations on graphs, notably Weisfeiler-Leman equivalence, can be characterised in terms of counting homomorphisms. 

Define a \emph{homomorphism} from a graph $G$ to a graph $H$ to be any function $h$ from vertices of $G$ to vertices of $H$ which preserves edges in one direction: if there is an edge from $v$ to $w$ in $G$, then there is an edge from $h(v)$ to $h(w)$ in $H$. We write $\homset(G,H)$ for the set of homomorphism from $G$ to $H$. A seminal result of  Lov\'asz~\cite[p.~326]{lovasz1967operations} says that two graphs are isomorphic if and only if they admit the same number of homomorphisms from every graph:
\begin{align*}
\underbrace{G \simeq H}_{\text{isomorphism}} \qquad \text{iff} \qquad  |\homset(F,G)|=|\homset(F,H)| \ \text{for every graph $F$.}
\end{align*}
In~\cite{groheDellRattan2018},
Grohe, Dell and Rattan show that if one restricts the  class of graphs from which $F$ is taken, then one gets various well-studied relaxations of isomorphism. For example, if we only count homomorphisms from graphs $F$ of treewidth at most $k$, then the arising equivalence relation on graphs is Weisfeiler-Leman equivalence of dimension $k$.  The result of~\cite{groheDellRattan2018},  that we use in this paper counts homomorphisms from paths.

\begin{theorem}\label{thm:grohe}(\cite[Theorem 2]{groheDellRattan2018})
    For all graphs $G_1,G_2$, the following  are equivalent:
    \begin{enumerate}
        \item for every $i \in \set{0,1,\ldots}$, if $P_i$ is the path of length $n$ then
        \begin{align*}
        |\homset(P_i,G_1) | = |\homset(P_i,G_2)|.
        \end{align*}
        \item $G_1$ and $G_2$  have the same number of vertices, say $n$, and there is an $n \times n$ square matrix $X$ with real coefficients such that:
        \begin{enumerate}
            \item each row in $X$ sums up to 1
            \item each column in $X$ sums up to 1
            \item $A_1 X = X A_2$, where $A_i$ is the incidence matrix of graph $G_i$.
        \end{enumerate}
    \end{enumerate}
\end{theorem}
Note that a homomorphism from a path of length $n$ is the same thing is a walk of length $n$, i.e.~a sequence of $n+1$ vertices (possibly with repetitions) interleaved with $n$ edges that connect them. We call a walk \emph{empty} if it contains no edges. In other words, condition 1 above says that for every $n$, the two graphs have the same number of walks of length $n$. In particular, if we take $n=0$, it implies that the graphs have the same number of vertices.     

Condition 2 in the above theorem can be seen as a relaxation of isomorphism in the following sense. 
If in condition 2 we add the requirement that all coefficients in $X$ are non-negative (real or rational, does not make a difference), then we get  \emph{fractional isomorphism}, which is characterised by homomorphisms from trees~\cite[Theorem 1]{groheDellRattan2018}. If we add the requirement that all coefficients in $X$ are non-negative integers, then we get isomorphism.  

\smallparagraph{Decidable equivalence, modulo counting homomorphisms from paths} 
In the following theorem, we show that equivalence for graph-to-graph \mso transductions of bounded treewidth is decidable, when output graphs are identified modulo the equivalence relation from Theorem~\ref{thm:grohe}. 
% We would have preferred to prove decidability of equivalence modulo isomorphism, but so far we have only managed to apply our technique to the equivalence relation from Theorem~\ref{thm:grohe}.
\begin{theorem}\label{thm:path-equivalence}
    Let $\sim$ be the equivalence relation on graphs described in Theorem~\ref{thm:grohe}. 
    The following problem is decidable:
    \decisionproblem{$\sim$-equivalence for graph-to-graph \mso transductions of bounded treewidth.}{ $\ell, k \in \set{1,2,\ldots}$ and two functional graph-to-graph \mso transductions $\Tt$.}{is it the case that for every input $G$ of treewidth at most $\ell$, the two outputs  $\Tt_1(G),\Tt_2(G)$ are equivalent under $\sim$ and have treewidth at most $k$?}
\end{theorem}
% By Lemma~\ref{lem:transduction-to-registers}, the above problem reduces to deciding $\sim$-functionality for nondeterministic \regTsover{\aalg}, where $\aalg$ is the algebra of $k$-sourced graphs. Here, $\sim$-functionality means that output graphs are identified modulo $\sim$. 
 
For the same reasons as in Fact~\ref{fact:equi-decidable}, the above problem is equi-decidable with the $\sim$-functionality problem, where one asks if a nondeterministic transduction can produce two $\sim$-inequivalent outputs on some input, under assumption of bounded treewidth. 

The rest of this section is devoted to showing decidability for the  $\sim$-equivalence problem. We first show how a register transducer can compute a representation of the output graph modulo $\sim$ in terms of a power series. Then we show that equivalence is decidable for register transducers which manipulate such power series. Combining this with the reduction to register transducers from Lemma~\ref{lem:transduction-to-registers}, we get the theorem.

\smallparagraph{Walk series}
A sequence of natural numbers $a_0,a_1,\ldots$ can be visualised as an infinite univariate polynomial
\begin{align*}
 A = a_0 + a_1x^1 + a_2x^2 + \cdots.
\end{align*}
We use the name \emph{power series} for such polynomials, and write $\powerseries{ \Nat}$ for the set of power series. 
We use the name $i$-th term for $a_i$; and we call $a_0$ the constant term. The power series form a semiring, with addition defined coordinatewise, and product being Cauchy product. Among all power series, we are particularly interested in the rational ones: a power series is called \emph{rational} if it can be generated from finite power series (a finite power series is one where all but finitely many terms are zero) using the semiring operations $+$ and $\times$ of power series, as well as the following unary  operation called \emph{Kleene plus}:
\begin{align*}
 A^+ \eqdef A + A^2 + A^3 + \cdots.
\end{align*}   
Kleene plus   is defined only when the constant term   in $A$ is zero, which ensures that the infinite sum in $A^+$ is well defined. Otherwise, if $A$ would have a nonzero constant term $a_0$, then its Kleene plus would need to have $a_0 + a_0^2 + a_0^3 + \cdots$ as its constant term.

For a graph $G$, define its \emph{walk series} to be the power series 
\begin{align*}
  a_1 x^1 + a_2x^2 + \cdots \qquad \text{where $a_i$ is the number of walks in $G$ of length $i \ge 1$.}
\end{align*}
Not how we count the lengths of nonempty walks; this is to ensure that the constant term will be zero.
By definition, two graphs with the same number of vertices are $\sim$-equivalent if and only if they have the same number of vertices and the same walk series; as mentioned in the introduction of this Section, computing number of vertices is straightforward and will be omitted later on. 
The first step in the proof of Theorem~\ref{thm:path-equivalence} is the following lemma, which says that the walk series is always a  rational power series, and can be computed by a register transducer.
\begin{lemma}\label{lem:compute-power-series}    
    Define  $\aalg$ to be the  algebra where
    \begin{description}
        \item[Domain:]power series in $\powerseries \Nat$ which are rational and have a  zero constant term;
        \item[Operations:] $+$, $\times$, Kleene plus and constants $1$ and $x$.
    \end{description}
    For every $k \in \set{1,2,\ldots}$ and every nondeterministic register transducer $\Aa$ over the algebra of $k$-sourced graphs, there is a  nondeterministic register transducer $\Bb$ over the algebra  $\aalg$ defined above which makes the following diagram commute (arrows are binary relations):
    \begin{align*}
    \xymatrix@C=1cm{
       &  \trees \Sigma   
        \ar[dr]^{\Bb}
        \ar[dl]_{\Aa} \\
        \txt{graphs of
        treewidth $< k$} \ar[rr]_-{G \mapsto \text{walk series of $G$}}& & \aalg
    }
    \end{align*}
\end{lemma}
\begin{proof}[Proof sketch] The main idea in the proof is the compositionality of walk series which is explained below. 
    An \emph{inner walk} in a $k$-sourced graph is defined to be a walk that does not visit sources, with the possible exception of the first and last vertex in the walk.  For $i,j \in \set{1,\ldots,k}$ define  $\varphi_{ij}(G)$ to be the power series where the $n$-th term is the number of nonempty inner walks that begin in the $i$-th source and end in the $j$-th source.     We now show how the series described above can be maintained compositionally  while applying the operations from the algebra of $k$-sourced graphs. 
    
      Consider first the fusion of two $k$-sourced graphs: an inner walk in the fusion is the same thing as an inner walk in either the first or second component of the fusion, and these cases are disjoint (because of parallel edges). Therefore, we have:
    \begin{align*}
    \varphi_{ij}(\text{fusion of $G$ and $H$}) =  \varphi_{ij}(G) + \varphi_{ij}(H).
    \end{align*}
    The more interesting case is the forget operation. Suppose that $G$ is a $k$-sourced graph and $G_\ell$ is the result of forgetting source $\ell \in \set{1,\ldots,k}$. An inner walk in $G_\ell$ is either an inner walk in $G$, or it can be decomposed as a composition of: first go to source $\ell$, then take a finite number of loops around source $\ell$, and then leave source $\ell$. Since composing walks corresponds to Cauchy product of power series, we get the  following equation:
    \begin{align*}
    \varphi_{ij}(\text{forget source $\ell$ in $G$}) = \varphi_{ij}(G) + \varphi_{i\ell}(G) \times (1 + (\varphi_{\ell \ell}(G))^+) \times \varphi_{\ell j}(G)
    \end{align*}
    
Using these compositionality ideas, we prove the lemma in the appendix.\end{proof}

\smallparagraph{Equivalence for transducers over rational power series}
Thanks to the above lemma, in order to prove Theorem~\ref{thm:path-equivalence}, it remains to show that equivalence is decidable for register transducers over algebra $\aalg$ of rational power series that is used in the above lemma.  Here the main idea is going to be the embedding of rational power series into rational functions. 

Define a \emph{rational function} to be a pair of polynomials in $\Int[x]$, written as $P/Q$, such that $Q \neq 0$, and modulo the equivalence relation which identifies $P_1/Q_1$ with $P_2/Q_2$ if $P_1Q_2 = P_2Q_1$. We write $\Int(x)$ for the rational functions; this is a field. One can embed the rational power series into rational functions as follows. A finite power series $A$ is mapped to itself (formally, it is mapped to a fraction where the enumerator is $A$ and the denumerator is $1$), and the map is extended to the entire semiring as follows:
\begin{align*}
\alpha(A \cdot B) = \alpha(A) \cdot \alpha(B) \quad \alpha(A + B) = \alpha(A) + \alpha(B) \quad \alpha(A^+) = \frac{\alpha(A)}{1-\alpha(A)}.
\end{align*}
One can show that $\alpha$ defined this way is a semiring homomorphism from  $\aalg$ to the field $\Int(x)$ of rational functions. Since all operations in the algebra $\aalg$ correspond to term operations in $\Int(x)$, it follows that equivalence of register transducers over the algebra $\aalg$ reduces to equivalence for register transducers over the field $\Int(x)$.

It remains to show that equivalence is decidable for register transducers over the field $\Int(x)$. This is true because $\Int(x)$ is a special case of a computable field: its elements can be enumerated in a way so that the field operations are computable.
\begin{theorem}\label{thm:equivalence-for-computable-field}
    For every computable field $\mathbb K$, 
    equivalence is decidable for register transducers over $\mathbb K$. 
\end{theorem}
This result is proved in the appendix, by adapting a proof for the special case of the field of rational numbers that was given in\cite[Theorem 6.6]{seidlManethKemper2018}. Applying the above result to the field $\Int(x)$, and then the encoding of $\aalg$ in $\Int(x)$, we see that equivalence is decidable for register transducers over $\aalg$. Combining this with Lemma~\ref{lem:compute-power-series}, we finish the proof of Theorem~\ref{thm:path-equivalence}.



