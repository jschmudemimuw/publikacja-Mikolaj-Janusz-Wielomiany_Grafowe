\subsection{Equivalence for register transducers over a computable field}\label{sec:pol-transducers-any-ring}

%(szkic wstępu) Equivalence of tree-to-($\Sigma^*, \cdot$) register transducers has been proved in \cite{seidlManethKemper2018}. The proof goes by simulating them with tree-to-\Z \emph{polynomial transducers}, which are a kind of register transducers that we define in next paragraph, and showing decidability of equivalence for them. To prove the main theorem we will use decidability of equivalence of tree-to-\Ring polynomial transducers, where \Ring is a ring of rational power series.

In this section, we give a proof of Theorem \ref{thm:equivalence-for-computable-ring}.
\thmequivalenceComputableRing*
%\begin{restatable}{corollaryrestat}{corEquivPolTransducersZrat}\label{cor:equivalence-pol-transducers-Zrat}
%		Equivalence is decidable for functional register transducers over both rings: $\Z[\enum x1n]$ and \Zrat.
%\end{restatable}
First we give definitions concerning computable rings. Later we sketch a proof from \cite[Theorem 6.6]{seidlManethKemper2018} of decidability of equivalence for register transducers over field \Q, from which follows the same results for \Z. Later we show that the same proof works for any computable field (Lemma \ref{lem:equivalence-for-computable-field}), from which Theorem \ref{thm:equivalence-for-computable-ring} follows.

\smallparagraph{Computable rings and fields}
By a ring we always mean a \emph{commutative} ring.
We say that a ring $\Ring$ is \emph{computable} if its elements can be enumerated in a way that addition, multiplication (and equality test, if the enumeration is not unique) are computable functions. %A \emph{nonzero} element $r$ of $\Ring$ is called a \emph{zero divisor} if $r \cdot s = 0$ for some $s \neq 0$.
We view field as an algebra, with addition, multiplication, and constants for every element; a field is computable if it is computable as a ring.

%Following \cite{seidlManethKemper2018}, by a \emph{polynomial transducer} we mean any register transducer over an algebra which is a ring.

\smallparagraph{The proof for \Z}
We sketch the proof of decidability of equivalence of \polTsover{\Q}, following \cite{seidlManethKemper2018}.

%Ring \Z can be embedded into field \Q and the proof is done for \polTsover{\Q}.
The proof goes by overapproximation, for each control state $q \in Q$, the set of configurations reachable in $q$ by a set of solutions of a system of polynomial equations (i.e. Zariski closed set).
$Q$-tuple of (polynomial) ideals that represent those systems forms an \emph{inductive invariant}.
The main algorithm runs two semiprocedures in parallel -- one searching for a counterexample to equivalence, one searching for an inductive invariant that certifies equivalence. The algorithm can be formulated for any ring, %\footnote{And for any algebra in fact. However, we do not discuss this further due to lack of place.}
but it works for field \Q for the following reasons.

The certificate of equivalence (inductive invariant) is a tuple of ideals that satisfy certain inclusions. Testing if those inclusions hold is done by Buchberger's algorithm. Every ideal can be finitely represented due to Hilbert's Basis Theorem and in consequence, tuples of them can be enumerated.
The proof of existence of inductive invariants (for a pair of equivalent polynomial transducers) uses \cite[Lemma 6.3]{seidlManethKemper2018}, let us denote it \emph{Product Lemma}, about ideals of polynomials that zero on cartesian products of sets.

\smallparagraph{Generalisation to other rings} Hilbert's Basis Theorem and Product Lemma hold for any field, with the same proof as for \Q (in fact, this is stated explicitly in \cite{seidlManethKemper2018} before Product Lemma). Buchberger's algorithm works for any computable field as well. We therefore obtain the following Lemma.

\begin{lemma}\label{lem:equivalence-for-computable-field}
Let $\Field$ be a computable field, considered as an algebra with addition and multiplication operations, with constants being the whole universe. Then equivalence is decidable for register transducers over $\Field$.
\end{lemma}

In consequence equivalence is decidable for any computable ring that can be embedded into a field. %; rational power series can be embedded into field of rational functions the following way.
%
%Let $A$ be a proper rational power series. Then $A^+ \cdot (1-A) = A$, hence $A^+$ can be seen as a fraction of polynomials of $A$; by induction, all proper rational power series are fractions of polynomials, which yields the embedding. Formally, a finite power series, i.e.~a polynomial in $\Int[x]$ can be seen as a rational function with denominator 1 and this mapping can be extended to a function $\alpha : \aalg \to \Int(x)$ on all proper rational series 
%which is defined by:
%\begin{align*}
%\alpha(A \times B) = \alpha(A) \times \alpha(B) \quad \alpha(A + B) = \alpha(A) + \alpha(B) \quad \alpha(A^+) = \frac{\alpha(A)}{1-\alpha(A)}.
%\end{align*}
%One can show that the above definition gives indeed a function, which is furthermore injective. By definition $\alpha$ is a semiring homomorphism, which finishes the construction of embedding. 
% Since all operations in the algebra $\aalg$ correspond to term operations in $\Int(x)$ augmented with division operation, it follows that equivalencse of register transducers over the algebra $\aalg$ reduces to equivalence for register transducers over the field $\Int(x)$, whose register updates may use division.
%Therefore we obtain the following result.
%%\begin{proof}
%%	Like every commutative ring without zero divisors, $\Ring$ embeds to a field (the embedding is an injective ring homomorphism). Since $\Ring$ is computable, then the field is also computable. 
%%\end{proof}
%From this the Corollary \ref{cor:equivalence-pol-transducers-Zrat} follows.
%\corEquivPolTransducersZrat*
%\begin{proof}%[Proof of Corollary \ref{cor:equivalence-pol-transducers-Zrat}]
%	This is an application of Therem \ref{thm:equivalence-polynomial-automata-over-a-ring} -- 
%It suffices to observe that both ring of polynomials and ring \Zrat are commutative and have no zero divisors.
%\end{proof}
%
%In fact, \Zrat can be embedded to \emph{some} field for a more abstract reason (the above construction is a proof of its computability though).
Such rings can be characterised the following way.
\begin{lemma}
\label{lem:ring-with-no-zero-divisors-subring-of-a-field}
		Let $\Ring$ be a ring. Then
		\begin{enumerate}[(i)]
			\item $\Ring$ can be embedded into a field if and only if it has no zero divisors,
			\item if $\Ring$ is computable then the embedding can be done effectively.
		\end{enumerate}
\end{lemma}
\begin{proof}
	(Computable) ring $\Ring$ with no zero divisors can be (effectively) embedded into its field of fractions.
\end{proof}
We conclude with the proof of Theorem \ref{thm:equivalence-for-computable-ring}.
\begin{proof}[Proof of Theorem \ref{thm:equivalence-for-computable-ring}]
	Combine Lemma \ref{lem:ring-with-no-zero-divisors-subring-of-a-field} with Lemma \ref{lem:equivalence-for-computable-field}.
\end{proof}
We finish this section with proof of Theorem \ref{thm:equivalence-with-division}.
\thmequivalenceWithDivision*
\begin{proof}
(fill-in proof)
\end{proof}
%\begin{remark}
%Product Lemma is used in the proof only if transducers' input alphabet contains a symbol of rank larger than 1. Therefore for \emph{string}-to-$\Ring$ transducers, Theorem generalises to rings that have finite representations of ideals and have decidable ideal inclusion -- there are examples of such rings, which are not a subring of some field.
%\end{remark}
