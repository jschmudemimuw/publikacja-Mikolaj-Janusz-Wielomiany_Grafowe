
\section{Equivalence for transducers with outputs of treewidth 1}
In this section, we consider transductions which output graphs of treewidth 1, i.e.~acyclic graphs. 

\begin{theorem} The following problem is decidable.
    \decisionproblem{equivalence for functional tree-to-acyclic-graph \mso transductions}{
         $\ell \in \set{1,2,\ldots}$ and two functional \mso graph-to-graph transductions $\Tt_1, \Tt_2$.
    }{is it the case that for every input $G$ of treewidth at most $\ell$, the outputs $\Tt_1(G)$ and $\Tt_2(G)$ are isomorphic acyclic graphs?}
\end{theorem}

The approach we take in this section is an extension of~\cite[Section 3.3]{boiretReducingTransducerEquivalence2018}, which had a similar result, except that both inputs and outputs were rooted acyclic graphs, i.e.~acyclic graphs with a distinguished vertex. 

\begin{center}
    (finish)
\end{center}

In the proof, we will reduce 
We view as a ring as an algebra with $+$ and $\times$, and constants for $0$ and $1$ (in fact, constants could be added for all elements of the ring). We say that a ring is \emph{computable} if its elements can be finitely represented so that the ring operations $+$ and $\times$ become computable. We say that a ring has  \emph{no zero divisors} if $a \times b$ implies $a = 0$ or $b=0$.


\begin{theorem}
    Functionality is decidable for tree-to-$\Ring$ register transducers, where $\Ring$ is any computable ring without zero divisors. 
\end{theorem}