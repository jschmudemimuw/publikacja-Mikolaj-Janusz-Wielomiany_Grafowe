
\section{Equivalence  modulo a certain equivalence relation}
In this section, we show that equivalence is decidable for graph-to-graph transducers of bounded treewidth, modulo a certain equivalence on the output graphs which is more relaxed than isomorphism.  

\smallparagraph{Counting homomorphisms} We begin by describing the equivalence on output graphs.
This equivalence is based on a paper by Grohe, Dell and Rattan~\cite{groheDellRattan2018}, which gives examples of how various equivalence relations on graphs, notably Weisfeiler-Leman equivalence, can be characterised in terms of counting homomorphisms. 

Define a \emph{homomorphism} from a graph $G$ to a graph $H$ to be any function $h$ from vertices of $G$ to vertices of $H$ which preserves edges: if there is an edge from $v$ to $w$ in $G$, then there is an edge from $h(v)$ to $h(w)$ in $H$. We write $\homset(G,H)$ for the set of homomorphism from $G$ to $H$. A seminal result of  Lov\'asz~\cite[p.~326]{lovasz1967operations} says that two graphs are isomorphic if and only if they admit the same number of homomorphisms from every graph:
\begin{align*}
\underbrace{G \simeq H}_{\text{isomorphism}} \qquad \text{iff} \qquad  |\homset(F,G)|=|\homset(F,H)| \ \text{for every graph $H$}
\end{align*}
Grohe, Dell and Rattan show that if one restricts the class of graphs from which the domain $F$ of the homomorphism is taken, then one obtained various well-studied relaxations of isomorphism. For example, if we only count homomorphisms for graphs $F$ of treewidth at most $k$, then the arising equivalence relation on graphs is Weisfeiler-Leman equivalence of dimension $k$.  Among the results of~\cite{groheDellRattan2018}, the one that we use in this paper is the following one, which counts homomorphisms from paths.

\begin{theorem}\label{thm:grohe}(\cite[Theorem 2]{groheDellRattan2018})
    For two graphs, the following conditions are equivalent:
    \begin{enumerate}
        \item for every $i \in \set{1,2,\ldots}$, if $P_i$ is the path of length $n$ then
        \begin{align*}
        |\homset(P_i,G_1) | = |\homset(P_i,G_2)|.
        \end{align*}
        \item there exists matrix $X$ with real coefficients such that:
        \begin{enumerate}
            \item the number of columns in $X$ is the number of vertices in $G_1$;
            \item the number of rows in $X$ is the number of vertices in $G_2$;
            \item each row in $X$ sums up to 1
            \item each column in $X$ sums up to 1
            \item $A_1 X = X A_2$, where $A_i$ is the incidence matrix of graph $G_i$.
        \end{enumerate}
    \end{enumerate}
\end{theorem}
In condition 2, if we add the requirement that all coefficients in $X$ are nonnegative (real or rational, does not make a difference), then we get  \emph{fractional isomorphism}, which is characterised by homomorphisms from trees~\cite[Theorem 1]{groheDellRattan2018}. If we add the requirement that all coefficients in $X$ are nonnegative integers, then we get isomorphism.  

\smallparagraph{Decidable equivalence, modulo counting homomorphisms from paths} 
In the following theorem, we show that equivalence for graph-to-graph \mso transductions of bounded treewidth is decidable, when output graphs are identified modulo the equivalence relation from Theorem~\ref{thm:grohe}. 
% We would have preferred to prove decidability of equivalence modulo isomorphism, but so far we have only managed to apply our technique to the equivalence relation from Theorem~\ref{thm:grohe}.
\begin{theorem}\label{thm:path-equivalence}
    Let $\sim$ be the equivalence relation on graphs described in Theorem~\ref{thm:grohe}. 
    The following problem is decidable:
    \begin{itemize}
        \item {\bf Input.} Numbers $\ell, k \in \set{1,2,\ldots}$ and  graph-to-graph \mso transductions $\Tt_1,\Tt_2$.
        \item {\bf Output.} Are the transductions equivalent modulo $\sim$, for inputs of treewidth $\ell$ and outputs of treewidth $k$, which means that
        \begin{align*}
        (G,H_1) \in \Tt_1 \text{ and } (G,H_2) \in \Tt_2 \quad \text{implies} \quad H_1 \sim H_2  \qquad \text{for all }\myunderbrace{G}{treewidth $\ell$} \text{ and } \underbrace{H_1,H_2}_{\text{treewidth $k$}}.
        \end{align*}
    \end{itemize}
\end{theorem}
 By (the reasoning behind) Corollary~\ref{cor:reduce-to-transducers}, the problem from the above theorem reduces to the following decision problem:
 \smallskip

\begin{itemize}
    \item {\bf Input.} A number $k \in \set{1,2,\ldots}$ and two register transducers over the algebra of $k$-sourced graphs, recognising functions
    \begin{align*}
    f_1,f_2 : \trees \Sigma \to \text{graphs of treewidth $\le k$}.
    \end{align*}
    \item {\bf Output.} Are the transducers $\sim$-equivalent, i.e.~equal inputs give $\sim$-equivalent outputs.
\end{itemize}
\smallskip
 
The rest of this section is devoted to showing that the above problem, about register transducers, is decidable. We first show how a register transducer can compute a representation of the output graph modulo $\sim$ in terms of a power series (Section~\ref{sec:power-series}), and then we show that equivalence is decidable for register transducers which manipulate such power series (Section~\ref{sec:decide-power-series}).

\subsection{Computing the power series for walks in a graph}
\label{sec:power-series}
Define a \emph{univariate formal power series over a semiring $\Ring$} to a sequence of semiring elements $a_0,a_1,\ldots$ which is represented as an infinite univariate polynomial
\begin{align*}
 A = a_0 + a_1x^1 + a_2x^2 + \cdots.
\end{align*}
We simply say power series, instead of univariate formal power series.
We use the name $i$-th term for $a_i$; and we call $a_0$ the constant term.
We write $\powerseries{\Ring}$ for the  power series over a semiring $\Ring$; this is a semiring. Among all power series, we are particularly interested in the rational ones: a power series is called \emph{rational} if it can be generated from finite power series (a finite power series is one where all but finitely many terms are zero) using the semiring operations $+$ and $\times$ of power series, as well as the following unary  operation called \emph{Kleene plus}:
\begin{align*}
A \quad \mapsto \quad A^+ \eqdef A + A^2 + A^3 + \cdots,
\end{align*}   
which  is defined only when the constant term   in $A$ is zero. The assumption on a zero constant term ensures that the infinite sum in $A^+$ is well defined: under this assumption, the $i$-th term in $A^j$ with $j>i$ is zero. 


The first step in the proof of Theorem~\ref{thm:path-equivalence} is the following lemma, which says that if we describe the number of homomorphisms from paths to a given graph using a power series, then this power series is rational, and furthermore it can be computed by a register transducer.
\begin{lemma}\label{lem:compute-power-series}    
    Let $\aalg$ be the algebra where  the domain is power series in $\powerseries \Nat$ that are rational and have a zero constant term, and the  operations are  $+$, $\times$ and Kleene plus.
For every $k \in \set{1,2,\ldots}$ there is an algebra homomorphism
\begin{align*}
\varphi : \text{algebra of $k$-sourced graphs} \to \aalg^{[k]}
\end{align*}
whose kernel, when restricted to graphs without sources, is exactly $\sim$, i.e.
\begin{align*}
\varphi(G_1)=\varphi(G_2) \text{ iff } G_1 \sim G_2 \qquad \text{for every graphs without sources $G_1$ and $G_2$}
\end{align*}

    For every $k \in \set{1,2,\ldots}$ there is a tree-to-$\aalg$ register transducer which does this:
    \begin{itemize}
        \item {\bf Input.} A width $k$ tree decomposition
        \item {\bf Output.} A rational power series 
        \begin{align*}
        a_1 x^1 + a_2 x^2 + \cdots 
        \end{align*}
        where $a_i$ is the number of length $i$ in the  underlying graph of the tree decomposition. 
    \end{itemize} 
\end{lemma}
\begin{proof}
    (fill in)
\end{proof}

\subsection{Equivalence for transducers that output rational power series}
\label{sec:decide-power-series}
\begin{lemma}
    Equivalence for register transducers over $\aalg$ is decidable, where $\aalg$ is the algebra from Lemma~\ref{lem:compute-power-series}. 
\end{lemma}




\begin{theorem}
    Let $\mathbb K$ be a ring which:
    \begin{itemize}
        \item is computable, i.e.~its elements can be finitely represented so that the ring operations $+$ and $\times$ become computable;
        \item has no zero divisors, i.e.~$a \times b$ implies $a = 0$ or $b=0$.
    \end{itemize}
    Then equivalence is decidable for tree-to-$\mathbb K$ register transducers.
\end{theorem}

\begin{lemma}
    There is a register automaton in 
\end{lemma}