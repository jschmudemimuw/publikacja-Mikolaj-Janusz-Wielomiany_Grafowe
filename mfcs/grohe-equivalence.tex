
\section{Equivalence  modulo a certain equivalence relation}
In this section, we present our main result, which says  that equivalence is decidable for graph-to-graph transducers of bounded treewidth, modulo a certain equivalence on the output graphs which is more relaxed than isomorphism.  

\smallparagraph{Counting homomorphisms} We begin by describing the equivalence on output graphs.
This equivalence is based on a paper by Grohe, Dell and Rattan~\cite{groheDellRattan2018}, which gives examples of how various equivalence relations on graphs, notably Weisfeiler-Leman equivalence, can be characterised in terms of counting homomorphisms. 

Define a \emph{homomorphism} from a graph $G$ to a graph $H$ to be any function $h$ from vertices of $G$ to vertices of $H$ which preserves edges in one direction: if there is an edge from $v$ to $w$ in $G$, then there is an edge from $h(v)$ to $h(w)$ in $H$. We write $\homset(G,H)$ for the set of homomorphism from $G$ to $H$. A seminal result of  Lov\'asz~\cite[p.~326]{lovasz1967operations} says that two graphs are isomorphic if and only if they admit the same number of homomorphisms from every graph:
\begin{align*}
\underbrace{G \simeq H}_{\text{isomorphism}} \qquad \text{iff} \qquad  |\homset(F,G)|=|\homset(F,H)| \ \text{for every graph $H$}
\end{align*}
In~\cite{groheDellRattan2018},
Grohe, Dell and Rattan show that if one restricts the  class of graphs from which the domain $F$ of the homomorphism is taken, then one gets various well-studied relaxations of isomorphism. For example, if we only count homomorphisms for graphs $F$ of treewidth at most $k$, then the arising equivalence relation on graphs is Weisfeiler-Leman equivalence of dimension $k$.  Among the results of~\cite{groheDellRattan2018}, the one that we use in this paper is the following one, which counts homomorphisms from paths.

\begin{theorem}\label{thm:grohe}(\cite[Theorem 2]{groheDellRattan2018})
    For two graphs, the following conditions are equivalent:
    \begin{enumerate}
        \item for every $i \in \set{1,2,\ldots}$, if $P_i$ is the path of length $n$ then
        \begin{align*}
        |\homset(P_i,G_1) | = |\homset(P_i,G_2)|.
        \end{align*}
        \item there graphs have the same number of vertices, say $n$, and there is an $n \times n$ square matrix $X$ with real coefficients such that:
        \begin{enumerate}
            \item each row in $X$ sums up to 1
            \item each column in $X$ sums up to 1
            \item $A_1 X = X A_2$, where $A_i$ is the incidence matrix of graph $G_i$.
        \end{enumerate}
    \end{enumerate}
\end{theorem}
Note that a homomorphism from a path of length $n$ is the same thing is a walk of length $n$, i.e.~a sequence of vertices connected by edges, possibly with repetitions.  In other words, condition 1 above says that for every $n$, the two graphs have the same number of walks of length $n$. The number of walks of length 1 is the number of vertices, and hence condition 1 implies that the graphs have the same number of vertices.     

Condition 2 in the above theorem can be seen as a relaxation of isomorphism in the following sense. 
If in condition 2 we add the requirement that all coefficients in $X$ are non-negative (real or rational, does not make a difference), then we get  \emph{fractional isomorphism}, which is characterised by homomorphisms from trees~\cite[Theorem 1]{groheDellRattan2018}. If we add the requirement that all coefficients in $X$ are non-negative integers, then we get isomorphism.  

\smallparagraph{Decidable equivalence, modulo counting homomorphisms from paths} 
In the following theorem, we show that equivalence for graph-to-graph \mso transductions of bounded treewidth is decidable, when output graphs are identified modulo the equivalence relation from Theorem~\ref{thm:grohe}. 
% We would have preferred to prove decidability of equivalence modulo isomorphism, but so far we have only managed to apply our technique to the equivalence relation from Theorem~\ref{thm:grohe}.
\begin{theorem}\label{thm:path-equivalence}
    Let $\sim$ be the equivalence relation on graphs described in Theorem~\ref{thm:grohe}. 
    The following problem is decidable:
    \decisionproblem{$\sim$-equivalence for functional graph-to-graph \mso transductions of bounded treewidth.}{ $\ell, k \in \set{1,2,\ldots}$ and functional graph-to-graph \mso transductions $\Tt_1,\Tt_2$.}{is it the case that for every input $G$ of treewidth at most $\ell$, the two outputs  $\Tt_1(G),\Tt_2(G)$ are $\sim$-equivalent and have treewidth at most $k$?}
\end{theorem}
% By Lemma~\ref{lem:transduction-to-registers}, the above problem reduces to deciding $\sim$-functionality for nondeterministic tree-to-$\aalg$ register transducers, where $\aalg$ is the algebra of $k$-sourced graphs. Here, $\sim$-functionality means that output graphs are identified modulo $\sim$. 
 
The rest of this section is devoted to showing decidability for the above $\sim$-functionality problem. We first show how a register transducer can compute a representation of the output graph modulo $\sim$ in terms of a power series (Section~\ref{sec:power-series}), and then we show that equivalence is decidable for register transducers which manipulate such power series (Section~\ref{sec:decide-power-series}). Combining this with the reduction to register transducers from Lemma~\ref{lem:transduction-to-registers}, we get the theorem.

\subsection{Computing the power series for walks in a graph}
\label{sec:power-series}
Define a \emph{univariate formal power series over a semiring $\Ring$} to be  sequence of semiring elements $a_0,a_1,\ldots$ which is represented as an infinite univariate polynomial
\begin{align*}
 A = a_0 + a_1x^1 + a_2x^2 + \cdots.
\end{align*}
We simply say power series, instead of univariate formal power series.
We use the name $i$-th term for $a_i$; and we call $a_0$ the constant term.
We write $\powerseries{\Ring}$ for the  power series over a semiring $\Ring$; this is a semiring. Among all power series, we are particularly interested in the rational ones: a power series is called \emph{rational} if it can be generated from finite power series (a finite power series is one where all but finitely many terms are zero) using the semiring operations $+$ and $\times$ of power series, as well as the following unary  operation called \emph{Kleene plus}:
\begin{align*}
 A^+ \eqdef A + A^2 + A^3 + \cdots,
\end{align*}   
which  is defined only when the constant term   in $A$ is zero. The assumption on a zero constant term ensures that the infinite sum in $A^+$ is well defined: under this assumption, the $i$-th term in $A^j$ with $j>i$ is zero. 


For a graph $G$, define its \emph{path series} to be the power series 
\begin{align*}
  a_1 x^1 + a_2x^2 + \cdots \qquad \text{where $a_i$ is the number of walks in $G$ of length $i$.}
\end{align*}
Almost by definition, two graphs are $\sim$-equivalent if and only if they have the same path series. 
The first step in the proof of Theorem~\ref{thm:path-equivalence} is the following lemma, which says  that the path series is always  rational, and can be computed by a register transducer.
\begin{lemma}\label{lem:compute-power-series}    
    Let $\aalg$ be the algebra where:
    \begin{description}
        \item[Domain:]power series in $\powerseries \Nat$ which are rational and have a  zero constant term;
        \item[Operations:] $+$, $\times$, Kleene plus and constants $1$ and $x$.
    \end{description}
    For every $k \in \set{1,2,\ldots}$ and every nondeterministic  \treetotreewidth{ k} register transducer $\Aa$, there is a nondeterministic tree-to-$\aalg$ register transducer $\Bb$ which makes the following diagram commute (arrows are binary relations):
    \begin{align*}
    \xymatrix@C=4cm{
        \trees \Sigma   
        \ar[dr]^{\Bb}
        \ar[d]_{\Aa} \\
        \txt{graphs of\\
        treewidth $\le k$} \ar[r]_{G \mapsto \text{path series of $G$}}& \aalg
    }
    \end{align*}
\end{lemma}
\begin{proof}
        The proof is essentially an adaptation of a weighted version of the usual ''DFA to regular expression conversion'', with state elimination corresponding to source forgetting.
    
    Denote set of source nodes by $\sources(G)$ and set of nonsource nodes by a symbol $\bullet$. Define $\varphi$ to map a sourced graph $G$ to a tuple of power series in the following way:
    $$
    \varphi: G \longmapsto (g_{i,j})_{i,j \in \allv{G}},
    $$
    where $g_{i,j}$ is the power series of set of walks from $i$ to $j$ which are positive length and whose interiors do not touch sources (if $j=\bullet$ we take the set of such walks from $i$ to any nonsource node). More precisely, $g_{i,j} = \sum_{n=1}^{\infty} a_n x^n$, where $a_n$ is the number of such walks of length $n$. Note that a particular $g_{i,j}$ is defined if all sources among $\{i,j\}$ are defined. A graph without sources is mapped to (a tuple with one element:) $g_{\bullet, \bullet}$ and obviously for graphs without sources $G, H$ we have $\varphi(G) = \varphi(H)$ if and only if $G \sim H$.
    
    Now let us investigate how $\varphi(\forget_k(G))$ and $\varphi(\join(G, H))$ can be reconstructed from $\varphi(G), \varphi(H)$.
    Observe that:
    \begin{align*}
    	\forget_k(G)_{i,j} &= g_{i,j} + g_{i,k}\cdot(1 + g_{k,k}^+)\cdot g_{k_j},\\
    	\join(G, H)_{i,j} &= g_{i,j} + h_{i,j} \ (- x \text{ if there is an edge } i\text{--}j)
    \end{align*}
    The substraction of $x$ in second equation serves for not counting length-1 walk $i$--$j$ twice, if it exists.
    
    The above equations hold, just as in ''DFA to regular expression'' case, because they are substitutions of equalities of sets of walks ($g_{i,j}$ can be obtained from set of walks by substituting $x$ for each edge), and substitution respects operations $+, \cdot, ^+$.
    
    A desired transducer works as follows: it inteprets input term bottom-up in algebra $\aalg^k$ according to above equations, keeping information about edges between sources in state.
\end{proof}

\subsection{Equivalence for transducers that output rational power series}
\label{sec:decide-power-series}
By Lemmas~\ref{lem:transduction-to-registers} and~\ref{lem:compute-power-series}, the  equivalence problem from Theorem~\ref{thm:path-equivalence} reduces to functionality for nondeterministic tree-to-$\aalg$ register transducers. In this section, we complete the proof the theorem by showing that the latter problem is decidable.
\begin{lemma}\label{lem:functionality-decidable-power-series} If $\aalg$ is the algebra from Lemma~\ref{lem:compute-power-series}, then functionality is decidable for nondeterministic tree-to-$\aalg$ register transducers. 
\end{lemma}
\begin{proof}
    (fill in)
\end{proof}

