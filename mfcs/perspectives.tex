\section{Perspectives}
In this section we would like to discuss future perspectives of this approach. First question is for what other graph polynomials or power series a similar reasoning can be applied. A first example is Tutte polynomial, denote it $T(G)$, a polynomial that counts many substructures in graphs of given "rank", e.g. spanning forests, colourings. Tutte polynomial of a graph outputted by MSO transduction can be computed in case underlying transducer performs $\join$ operation along at most one vertex at a time, due to recurrences $T(G \cup H) = T(G) \cdot T(H), T(G\sqcup_a H) = T(G) \cdot T(G)$. The distinguishing power of Tutte polynomial is however very low in our context: all trees with $n$ edges have the same Tutte polynomial ($x^n$), hence it is not suitable for deciding equivalence on classes defined by treewidth.

Another examples is characteristic polynomial. Interestingly, two graphs have the same characteristic polynomial if and only if they have the same number of length-$n$ closed walks for each $n$ (cf. beginning of Section \ref{sec:equivalence-modulo}). It can be computed, again under assumption of transducer fusing along at most one common vertex at a time in the following way. For a sourced graph $G$, define $\varphi(G) = ({\widetilde{G}}^a, \overline{G}^a)_{a \in \sources(G)}$ to be a following "split" of characteristic polynomial: consider Laplace expansion of $A_G-t\cdot I$ at row corresponding to vertex $a$
$$
	\det(A_G-t\cdot I) = (-t)\cdot \underbrace{\det(A_{G\setminus a} - t\cdot I)}_{{\widetilde{G}}^a} + \underbrace{\text{ rest }}_{\overline{G}^a}.
$$
Now take two graphs $G$ with a common vertex $a$. If we order vertices of $G$ and $H$ so that $a$ is the last vertex of $G$ and the first vertex of $H$, adjacency matrix of $G\sqcup_aH$ looks the following way:
$$
	(\text{fill-in pic1})
$$
and its characteristic polynomial, computed by Laplace expansion at row corresponding to $a$ is the following:
\begin{multline*}
	\det(A_{G\sqcup_a H}) = \sum_{v \in (G- a)} G^1_v \cdot (\text{fill-in det1}) + (-t) \cdot (\text{fill-in det2}) + \sum_{w \in (H- a)} H^1_w \cdot (\text{fill-in det3}).
\end{multline*}
Now, by the fact that $\begin{vmatrix}
A & \ast\\
0 & B
\end{vmatrix}
=
\begin{vmatrix}
A & 0 \\
\ast & B
\end{vmatrix}
=
|A|\cdot|B|
$
(assume $A,B$ are square matrices, possibly of different size)
we get it is equal to
$$
	{\widetilde{G}}^a\cdot\overline{H}^a + (-t) \cdot {\widetilde{G}}^a \cdot {\widetilde{H}}^a + \overline{G}^a \cdot {\widetilde{H}}^a.
$$
The same proof and computation of other graph invariants (graph Laplacians) after $\join$ operation (called \emph{interface gluing} in the paper) can be found in \cite{contrerasGluingLaplacians20}.
The distinguishing power of characteristic polynomial is low in our context -- a result \cite{schwenkCospectral73} due to Schwenk says that almost all trees, in a sense of probability, are not uniquely characterized by their characteristic polynomial.

Second direction of further research might be to search for classess of graphs where some polynomials or power series have high distinguighing power. Even for the class of edge-labelled cycles, which can be encoded as graphs pathwidth 4 and treewidth 2, %dodając 1 lub 2widełki w każdym wierzchołku cyklu, w zależności czy ma etykiete a czy b
the question whether edge-labelled version of walk power series (i.e. $\sum_{n,m=0}^{\infty}\eta_{n,m} a^n b^m$, where $\eta_{n,m}$ equals number of walks with $n$ $a$-edges and $m$ $b$-edges) uniquely determines the graph is open.
%Jedyna literatura warta uwagi, którą udało mi się znaleźć to "Negami, Polynomial Invariants of Graphs" oraz "Polynomial invariants of graphs II Seiya Negami & Katsuhiro Ota " Planuję je przeczytać.